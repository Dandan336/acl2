\documentclass{beamer}
\usecolortheme{seahorse}
\usepackage{inconsolata}
\usepackage{alltt}

\beamertemplatenavigationsymbolsempty

\setbeamertemplate{footline}[frame number]

%%%%%%%%%%%%%%%%%%%%%%%%%%%%%%%%%%%%%%%%%%%%%%%%%%%%%%%%%%%%%%%%%%%%%%%%%%%%%%%%

\newcommand{\code}[1]{\texttt{#1}}
\newenvironment{codeblock}{\begin{alltt}}{\end{alltt}}

\newcommand{\bookpath}[1]{\textbf{\code{#1}}}

\newcommand{\newlibtitle}{\frametitle{New Libraries}}
\newcommand{\implibtitle}{\frametitle{Improved Libraries}}

%%%%%%%%%%%%%%%%%%%%%%%%%%%%%%%%%%%%%%%%%%%%%%%%%%%%%%%%%%%%%%%%%%%%%%%%%%%%%%%%

\begin{document}

%%%%%%%%%%%%%%%%%%%%%%%%%%%%%%%%%%%%%%%%%%%%%%%%%%%%%%%%%%%%%%%%%%%%%%%%%%%%%%%%

\title{What's New in the Community Books}

\subtitle{Since the ACL2-2017 Workshop}

\author{Cuong Chau\inst{4},
        Alessandro Coglio\inst{1},
        Jared Davis\inst{2},
        Shilpi Goel\inst{3},
        \\
        Matt Kaufmann\inst{4},
        Keshav Kini\inst{5},
        Mihir Mehta\inst{4},
        \\
        J Moore\inst{4},
        David Russinoff\inst{6},
        Julien Schmaltz\inst{7},
        \\
        Rob Sumners\inst{3},
        Sol Swords\inst{3}}

\institute{\inst{1}Kestrel Institute,
           \inst{2}Apple,
           \inst{3}Centaur,
           \inst{4}UT Austin,
           \inst{5}Oracle,
           \inst{6}ARM,
           \inst{7}???}

\date{ACL2-2018 Workshop}

\frame[noframenumbering,plain]{\titlepage}

%%%%%%%%%%%%%%%%%%%%%%%%%%%%%%%%%%%%%%%%%%%%%%%%%%%%%%%%%%%%%%%%%%%%%%%%%%%%%%%%

\begin{frame}

\newlibtitle

\bookpath{kestrel/apt}:
APT (Automated Program Transformations),
a toolkit
to transform programs and program specifications with automated support.
\begin{itemize}
\item
Includes two of Kestrel's $\sim$40 transformations.
\item
Also includes some utilities used across transformations.
\item
More forthcoming.
\end{itemize}

\

\bookpath{kestrel/java}:
A library for Java.
\begin{itemize}
\item
AIJ (ACL2 In Java), the deep embedding of ACL2 in Java
described in a paper at this Workshop.
\item
ATJ (ACL2 To Java), the Java code generator for ACL2
described in a paper at this Workshop.
\end{itemize}

\end{frame}

%%%%%%%%%%%%%%%%%%%%%%%%%%%%%%%%%%%%%%%%%%%%%%%%%%%%%%%%%%%%%%%%%%%%%%%%%%%%%%%%

\begin{frame}

\newlibtitle

\bookpath{kestrel/bitcoin}:
A (small start towards a) library for the Bitcoin cryptocurrency and ecosystem.
\begin{itemize}
\item
Executable specification of Base58 encoding and decoding.
\item
Executable specification of Base58Check encoding.
\end{itemize}

\

\bookpath{kestrel/ethereum}:
A library for the Ethereum
cryptocurrency and ecosystem.
\begin{itemize}
\item
Executable specification of RLP (Recursive Length Prefix) encoding;
declarative specification of RLP decoding.
\item
Executable specification of hex-prefix encoding.
\item
Kestrel is actively working on this.
\end{itemize}

\end{frame}

%%%%%%%%%%%%%%%%%%%%%%%%%%%%%%%%%%%%%%%%%%%%%%%%%%%%%%%%%%%%%%%%%%%%%%%%%%%%%%%%

\begin{frame}[fragile]

\newlibtitle

\bookpath{kestrel/utilities/digits-any-base}:
Conversions between natural numbers
and their representations in arbitrary bases.
\begin{itemize}
\item
Big and little endian.
\item
Minimal, minimal non-zero, or specified length.
\item
Several theorems, e.g.\ about inversions.
\end{itemize}

\

\bookpath{kestrel/utilities/xdoc}:
XDOCumentation utilities.
\begin{itemize}
\item
Constructors of well-tagged XDOC strings, e.g.
\begin{codeblock}
(xdoc::p "This is a paragraph.")
(xdoc::ul
  (xdoc::li "First unordered item.")
  (xdoc::li "Second unordered item."))
\end{codeblock}
\item
\code{defxdoc+} extends \code{defxdoc} with
\code{:order-subtopics t/nil} and \code{:default-parent t/nil}.
\item
These are candidate extensions of the \code{xdoc} library.
\end{itemize}

\end{frame}

%%%%%%%%%%%%%%%%%%%%%%%%%%%%%%%%%%%%%%%%%%%%%%%%%%%%%%%%%%%%%%%%%%%%%%%%%%%%%%%%

\begin{frame}

\newlibtitle

\bookpath{kestrel/utilities/fixbytes}:
Fixtypes for unsigned and signed bytes, and true lists thereof.
\begin{itemize}
\item
Macros to create fixtypes and theorems for a specified size.
The size may be a constrained nullary function,
e.g.\ useful to formalize C bytes.
\item
Several instances available; just include the respective file(s).
\item
These are candidate extensions of the \code{fty} library.
\end{itemize}

\

\bookpath{kestrel/utilities/integer-range-*.lisp}:
Utilities related to \code{integer-range-p}.
\begin{itemize}
\item
Parameterized recognizer \code{integer-range-listp}.
\item
Parameterized fixers
\code{integer-range-fix} and \code{integer-range-list-fix}.
\item
Several theorems.
\end{itemize}

\end{frame}

%%%%%%%%%%%%%%%%%%%%%%%%%%%%%%%%%%%%%%%%%%%%%%%%%%%%%%%%%%%%%%%%%%%%%%%%%%%%%%%%

\begin{frame}

\newlibtitle

\bookpath{kestrel/utilities/messages.lisp}:
A few utilities for \code{msgp} values,
e.g.\ to convert the first character to upper/lower case.

\

\bookpath{kestrel/utilities/untranslate-preprocessing.lisp}:
A macro \code{add-const-to-untranslate-preprocess}
to keep a named constant unexpanded in the screen output.

\

\bookpath{kestrel/utilities/symbols.lisp}:
Some utilities for symbols.
\begin{itemize}
\item
These could become a new \code{std/symbols} library.
\end{itemize}

\

\bookpath{kestrel/utilities/system/paired-names.lisp}:
Utilities for names consisting of
two parts with a customizable separator in between.
(Used by APT, but more general.)

\end{frame}

%%%%%%%%%%%%%%%%%%%%%%%%%%%%%%%%%%%%%%%%%%%%%%%%%%%%%%%%%%%%%%%%%%%%%%%%%%%%%%%%

\begin{frame}

\newlibtitle

TODO: add more contributions

\end{frame}

%%%%%%%%%%%%%%%%%%%%%%%%%%%%%%%%%%%%%%%%%%%%%%%%%%%%%%%%%%%%%%%%%%%%%%%%%%%%%%%%

\begin{frame}

\implibtitle

\bookpath{kestrel/soft}:
SOFT (Second-Order Functions and Theorems),
macros to mimic second-order functions and theorems.
\begin{itemize}
\item
Added full support for \code{defun-sk2}.
\item
Improved user interface.
\end{itemize}

\

\bookpath{kestrel/utilities/osets.lisp}:
Utilities for osets.
\begin{itemize}
\item
Added a fixtype for osets.
\item
These are candidate extensions of the \code{std/osets} library.
\end{itemize}

\

\bookpath{kestrel/utilities/error-checking.lisp}:
Utilities to check error conditions and return customizable error messages.
\begin{itemize}
\item
Improved the \code{def-error-checker} macro,
e.g.\ to support logic-mode error-checking functions.
\item
Added several error-checking functions.
\end{itemize}

\end{frame}

%%%%%%%%%%%%%%%%%%%%%%%%%%%%%%%%%%%%%%%%%%%%%%%%%%%%%%%%%%%%%%%%%%%%%%%%%%%%%%%%

\begin{frame}

\implibtitle

\bookpath{kestrel/system/world-queries.lisp}:
Utilities to query worlds.
\begin{itemize}
\item
Added and improved several utilities.
\item
There are two variants for most of these utilities:
a ``fast'' one and a ``logic-friendly'' one
(see documentation for details).
\item
These could become part of a new \code{std/system} library.
\end{itemize}

\

\bookpath{kestrel/utilities/defun-sk-queries.lisp}:
Utilities to query \code{defun-sk} functions.
\begin{itemize}
\item
Added support for the recently added \code{:constrain} option.
\item
These could become part of a new \code{std/system} library.
\end{itemize}

\end{frame}

%%%%%%%%%%%%%%%%%%%%%%%%%%%%%%%%%%%%%%%%%%%%%%%%%%%%%%%%%%%%%%%%%%%%%%%%%%%%%%%%

\begin{frame}

\implibtitle

\bookpath{kestrel/system/terms.lisp}:
Utilities to manipulate terms.
\begin{itemize}
\item
Added and improved several utilities.
\item
Moved some utilities to a separate file
\code{term-function-recognizers.lisp}.
\item
These could become part of a new \code{std/system} library.
\end{itemize}

\

\bookpath{kestrel/utilities/user-interface.lisp}:
Utilities for customizing screen output of user-defined events.
\begin{itemize}
\item
Added several utilities.
\end{itemize}

\end{frame}

%%%%%%%%%%%%%%%%%%%%%%%%%%%%%%%%%%%%%%%%%%%%%%%%%%%%%%%%%%%%%%%%%%%%%%%%%%%%%%%%

\begin{frame}

\implibtitle

\bookpath{kestrel/utilities/...}:
Started refactoring some of these utilities to reduce book dependencies.

\

\bookpath{misc/assert.lisp} \& \bookpath{misc/eval.lisp}:
Testing utilities.
\begin{itemize}
\item
Added some utilities moved from \code{kestrel/utilities/testing.lisp}.
\item
Added some XDOCumentation.
\item
Renamed some utilities for greater uniformity
(deprecated the old names).
\item
Reduced book dependencies.
\end{itemize}

\end{frame}

%%%%%%%%%%%%%%%%%%%%%%%%%%%%%%%%%%%%%%%%%%%%%%%%%%%%%%%%%%%%%%%%%%%%%%%%%%%%%%%%

\begin{frame}

\implibtitle

\bookpath{projects/x86isa}:
X86ISA, the formal model of the x86 ISA.
\begin{itemize}
\item
Added support for 32-bit mode; see paper at this Workshop.
\item
Improved and extended some documentation.
\item
TODO: add more
\end{itemize}

\end{frame}

%%%%%%%%%%%%%%%%%%%%%%%%%%%%%%%%%%%%%%%%%%%%%%%%%%%%%%%%%%%%%%%%%%%%%%%%%%%%%%%%

\begin{frame}

\implibtitle

\bookpath{projects/async}:
ASYNC, the framework for modeling and verifying the functional correctness
of asynchronous (self-timed) circuit models.
\begin{itemize}
\item
Developed a new compositional methodology for scalable formal
verification of functional properties of self-timed circuit designs.
\item
Verified the functional correctness of data-loop-free self-timed
circuits (see fifo/).
\item
Verified the functional correctness of a self-timed serial
adder/subtractor model (see serial-adder/).
\item
Verified the functional correctness of iterative self-timed circuit
models that compute the greatest-common-divisor (GCD) (see gcd/).
\item
Verified the functional correctness of self-timed circuits performing
arbitrated merge operations (see arbitration/).
\end{itemize}

\end{frame}

%%%%%%%%%%%%%%%%%%%%%%%%%%%%%%%%%%%%%%%%%%%%%%%%%%%%%%%%%%%%%%%%%%%%%%%%%%%%%%%%

\begin{frame}

\implibtitle

\bookpath{std/io/combine.lisp}:
Byte-combining libraries.
\begin{itemize}
\item
Added invertibility theorems for \code{combine16u} and \code{combine32u}.
\item
TODO: make these invertibility theorems compatible with \code{part-select}.
\item
TODO: prove similar theorems for \code{combine64u} as well as for the
signed-integer functions, \code{combine16s} et al.
\end{itemize}

\

\bookpath{centaur/bitops/rotate.lisp}:
Bit-vector rotation libraries.
\begin{itemize}
\item
Generalized existing theorems and added a new theorem for compositions
of \code{rotate-left} operations, as well as a theorem for
compositions of \code{rotate-right} operations.
\item
TODO: Add theorems for compositions of \code{rotate-left} and
\code{rotate-right} with each other.
\end{itemize}

\end{frame}

%%%%%%%%%%%%%%%%%%%%%%%%%%%%%%%%%%%%%%%%%%%%%%%%%%%%%%%%%%%%%%%%%%%%%%%%%%%%%%%%

\begin{frame}

\implibtitle

\bookpath{kestrel/utilities/strings/*}:
String manipulation libraries.
\begin{itemize}
\item Added several new rewrite rules.
\end{itemize}

\

\bookpath{projects/filesystems/*}:
Formal models of filesystems.
\begin{itemize}
\item
\texttt{M1} and \texttt{M2}, new filesystem models for FAT32,
described in a paper at this Workshop.
\end{itemize}
\end{frame}

%%%%%%%%%%%%%%%%%%%%%%%%%%%%%%%%%%%%%%%%%%%%%%%%%%%%%%%%%%%%%%%%%%%%%%%%%%%%%%%%

\begin{frame}

\implibtitle

TODO: add more contributions

\end{frame}

%%%%%%%%%%%%%%%%%%%%%%%%%%%%%%%%%%%%%%%%%%%%%%%%%%%%%%%%%%%%%%%%%%%%%%%%%%%%%%%%

\begin{frame}

\frametitle{Additional Contributions}

\bookpath{workshop/references}:
BibTeX references for all the ACL2 Workshop papers,
and a LaTeX document that shows them.
TODO: add/improve as needed

\

\textbf{Developers Guide}:
Several XDOCumentation pages intended for ACL2 developers.
TODO: add/improve as needed

\end{frame}

%%%%%%%%%%%%%%%%%%%%%%%%%%%%%%%%%%%%%%%%%%%%%%%%%%%%%%%%%%%%%%%%%%%%%%%%%%%%%%%%

\end{document}
