% GENERAL FORMAT OF EACH SLIDE
% ----------------------------

%\begin{frame}

%\newlibtitle or \implibtitle % for new vs. improved libraries

%\bookpath{path/to/the/library}: % directory or file, use macro for font
% Short description of what this library is, 1-2 lines.
%\begin{itemize}
%\item
% A highlight about this new or improved library.
%\item
% Another highlight.
%\item
% Say if it's described in a paper at this workshop,
% in which case, there's little to say here at all.
%\end{itemize}

%\separation % to separate the next entry, if any

%\bookpath{another/path}: % as above

%\end{frame}

% ORDER OF THE SLIDES AND ENTRIES
% -------------------------------

% First all the new libraries, in alphabetical order.

% Then all the improved libraries, in alphabetical order.

% Then one or more slides with additional contributions.

% Don't worry about adding your contributions in order:
% Alessandro will quickly put them in order if needed,
% and also fix any formatting issues.

% TITLE SLIDE
% -----------

% Add your name as co-author, in alphabetical order of last name.

% Add your organization, in alphabetical order.
% Renumber the \inst{n} instances as needed
% (both inside \author{...} and inside \institute{...}.

% OTHER REMARKS
% -------------

% If you want to show code block snippets,
% add [fragile] just after \begin{frame}.

% For guidelines on content and level of detail,
% please see existing entries.
% The goal is to provide enough information to get possible users interested.

%%%%%%%%%%%%%%%%%%%%%%%%%%%%%%%%%%%%%%%%%%%%%%%%%%%%%%%%%%%%%%%%%%%%%%%%%%%%%%%%

\documentclass{beamer}
\usetheme{Madrid}
\usecolortheme{crane}
\usepackage{inconsolata}
\usepackage{alltt}

\beamertemplatenavigationsymbolsempty

\setbeamertemplate{footline}[frame number]

%%%%%%%%%%%%%%%%%%%%%%%%%%%%%%%%%%%%%%%%%%%%%%%%%%%%%%%%%%%%%%%%%%%%%%%%%%%%%%%%

\newcommand{\code}[1]{\texttt{#1}}
\newenvironment{codeblock}{\begin{alltt}}{\end{alltt}}

\newcommand{\bookpath}[1]{\textbf{\code{#1}}}

\newcommand{\newlibtitle}{\frametitle{New Libraries}}
\newcommand{\implibtitle}{\frametitle{Improved Libraries}}

\newcommand{\separation}{\vspace*{1.5ex}}

%%%%%%%%%%%%%%%%%%%%%%%%%%%%%%%%%%%%%%%%%%%%%%%%%%%%%%%%%%%%%%%%%%%%%%%%%%%%%%%%

\begin{document}

%%%%%%%%%%%%%%%%%%%%%%%%%%%%%%%%%%%%%%%%%%%%%%%%%%%%%%%%%%%%%%%%%%%%%%%%%%%%%%%%

\title{What's New in the Community Books}

\subtitle{Since the ACL2-2018 Workshop}

\author{Cuong Chau\inst{1},
        Alessandro Coglio\inst{3} (presenter), \\
        Shilpi Goel\inst{2},
        Eric McCarthy\inst{3},
        Mihir Mehta\inst{5}, \\
        Yan Peng\inst{4},
        David Russinoff\inst{1},
        Eric Smith\inst{3}, \\
        Sol Swords\inst{2},
        Mertcan Temel\inst{5},
        Stephen Westfold\inst{3}}

\institute{\inst{1}ARM,
           \inst{2}Centaur,
           \inst{3}Kestrel Institute, \\
           \inst{4}University of British Columbia, \\
           \inst{5}University of Texas at Austin}

\date{ACL2-2020 Workshop}

\frame[noframenumbering,plain]{\titlepage}

%%%%%%%%%%%%%%%%%%%%%%%%%%%%%%%%%%%%%%%%%%%%%%%%%%%%%%%%%%%%%%%%%%%%%%%%%%%%%%%%

\begin{frame}

\newlibtitle

\bookpath{centaur/fgl}:
Bitblasting rewriter, successor to GL.
\begin{itemize}
\item A rewriter that does bitblasting, rather than a bitblaster that has a rewriter.
\item More extensible and customizable than GL.
\item Adds incremental SAT support.
\item Some fancy rewriter features discussed in the paper \textit{New Rewriter Features in FGL} at this workshop.
\end{itemize}

\end{frame}

%%%%%%%%%%%%%%%%%%%%%%%%%%%%%%%%%%%%%%%%%%%%%%%%%%%%%%%%%%%%%%%%%%%%%%%%%%%%%%%%

\begin{frame}

\newlibtitle

\bookpath{centaur/meta}:
New library of metareasoning utilities.
\begin{itemize}
\item Based on \bookpath{clause-processors/pseudo-term-fty.lisp}, treats pseudo-term as an FTY sum type.
\item Unification, substitution, variable collection.
\item Unconditional rewriter.
\item Let-abstraction algorithm.
\item Utilities for understanding rewrite, equivalence, and congruence rules.
\item Term measure that decreases under beta-reduction.
\item Utility that uses meta-extract plus a runtime check to
  effectively extend the functions understood by an evaluator.
\end{itemize}

\end{frame}

%%%%%%%%%%%%%%%%%%%%%%%%%%%%%%%%%%%%%%%%%%%%%%%%%%%%%%%%%%%%%%%%%%%%%%%%%%%%%%%%

\begin{frame}

\newlibtitle

\bookpath{centaur/svl}:
A framework to simulate and reason about Verilog modules with design hierarchy.
\begin{itemize}
\item Uses \code{centaur/sv} and \code{centaur/vl} books to generate
  simulation-ready SVL designs. May sometimes be used in lieu of SVTV.
\item Selected submodules may not be flattened to retain design hierarchy.
\item Does not support combinational loops (e.g., latches).
\end{itemize}

\end{frame}

%%%%%%%%%%%%%%%%%%%%%%%%%%%%%%%%%%%%%%%%%%%%%%%%%%%%%%%%%%%%%%%%%%%%%%%%%%%%%%%%

\begin{frame}

\newlibtitle

\bookpath{kestrel/crypto}:
Executable specifications and abstract interfaces of cryptographic functions.
\begin{itemize}
\item
Specification of short Weierstrass elliptic curves in a prime field.
Proof of closure under the group operation.
Instantiation to the secp256k1 curve used by Bitcoin and Ethereum.
\item
Specification of Deterministic ECDSA
(Elliptic Curve Digital Signature Algorithm)
following IETF RFC 6979.
\item
Specification of HMAC (Hashed Key Message Authentication Code)
following IETF RFC 2104.
Instantiated with SHA-256 and SHA-512.
\item
Specification of PBKDF2 (Password-Based Key Derivation Function 2)
following IETF RFC 8018.
Instantiated with HMAC-SHA-512.
\item
Specification of the Keccak family of permutations, sponge functions,
and hash functions, and the FIPS 202 SHA-3 hash functions based on Keccak.
\item
SHA-2.
\item
Padding.
\end{itemize}

\end{frame}

%%%%%%%%%%%%%%%%%%%%%%%%%%%%%%%%%%%%%%%%%%%%%%%%%%%%%%%%%%%%%%%%%%%%%%%%%%%%%%%%

\begin{frame}

\newlibtitle

\bookpath{kestrel/event-macros}:
Utilities to develop event macros (i.e.\ macros at the event level)
more quickly and consistently.
\begin{itemize}
\item
Validation and elaboration of inputs of event macros.
\item
Handling of applicability conditions
(i.e.\ theorems that must be proved for the event macro to apply).
\item
Controlling of screen output of event macros.
\item
XDOC constructors for both user and developer documentation of event macros.
\end{itemize}

\end{frame}

%%%%%%%%%%%%%%%%%%%%%%%%%%%%%%%%%%%%%%%%%%%%%%%%%%%%%%%%%%%%%%%%%%%%%%%%%%%%%%%%

\begin{frame}

\newlibtitle

\bookpath{kestrel/std/basic}:
An extension of the Std/basic library.
\begin{itemize}
\item
Added several functions to manipulate symbols.
\item
Added \code{mbt\$}, a variant of \code{mbt}
that requires non-\code{nil} instead of \code{t}.
\item
These will be gradually moved to \code{std/basic}.
\end{itemize}

\separation

\bookpath{kestrel/std/system}:
A new Std/system library
with standard system utilities
that complement the built-in ones.
\begin{itemize}
\item
Several utilities have been moved here from \code{kestrel/utilities/system},
and improved in the process.
\item
New utilities have been added.
\item
This is being gradually moved to \code{std/system}.
\end{itemize}

\end{frame}

%%%%%%%%%%%%%%%%%%%%%%%%%%%%%%%%%%%%%%%%%%%%%%%%%%%%%%%%%%%%%%%%%%%%%%%%%%%%%%%%

\begin{frame}

\newlibtitle

\bookpath{kestrel/std/util}:
An extension of the Std/util library.
\begin{itemize}
\item
Added \code{defarbrec},
to introduce recursive functions without proving termination right away.
Compared to similar existing tools, it is mainly aimed at use with APT.
\item
Added \code{deffixer}, to introduce fixers and associated theorems.
\item
Added \code{defiso}, to verify and record isomorphic mappings;
described in the paper
\textit{Isomorphic Data Type Transformations} at this workshop.
\item
Added \code{defsurj}, to verify and record surjective mappings.
\item
Added \code{defmax-nat},
to declaratively define the maximum of
a (possibly infinite) set of natural numbers.
\item
Added \code{defmacro+}, which enhances \code{defmacro} with XDOC integration.
\item
These will be gradually moved to \code{std/util}.
\end{itemize}

\end{frame}

%%%%%%%%%%%%%%%%%%%%%%%%%%%%%%%%%%%%%%%%%%%%%%%%%%%%%%%%%%%%%%%%%%%%%%%%%%%%%%%%

\begin{frame}

\newlibtitle

\bookpath{kestrel/utilities/obags}:
Ordered bags, i.e.\ finite bags represented as non-strictly ordered lists.
Cf.\ omaps (below) and osets (in Std/osets).

\separation

\bookpath{kestrel/utilities/omaps}:
Ordered maps, i.e.\ finite maps represented as strictly ordered alists.
Cf.\ obags (above) and osets (in Std/osets).

\separation

\bookpath{kestrel/utilities/orelse.lisp}:
Improved \code{orelse*}.

\separation

\bookpath{kestrel/utilities/sublis-expr-plus.lisp}:
Added \code{sublis-expr+},
which replaces terms by variables also inside lambda expressions.

\end{frame}

%%%%%%%%%%%%%%%%%%%%%%%%%%%%%%%%%%%%%%%%%%%%%%%%%%%%%%%%%%%%%%%%%%%%%%%%%%%%%%%%

\begin{frame}

\newlibtitle

\bookpath{projects/rp-rewriter}:
A customized rewriter and a verified
clause processor. Implemented for multiplier verification but it is a
generic rewriter. Details are discussed in the paper {\it RP-Rewriter: An
Optimized Rewriter for Large Terms in ACL2} at this workshop.

\separation

\bookpath{projects/rp-rewriter/lib/\{mult,mult2\}}:
An automated and efficient tool to verify integer multipliers.
\begin{itemize}
\item
Supports Booth Encoding,
simple partial products,
and Wallace-tree like multipliers.
\item Uses SVL and RP-Rewriter.
\item Two libraries, same  functionality.
  \begin{itemize}
  \item \code{mult}: more  user-friendly debugging.
  \item\code{mult2}: 30\% faster.
  \end{itemize}
\item  Very efficient:  64x64  multipliers  in less  than  2 seconds,  some
  1024x1024 multipliers in less than 10 minutes.
\end{itemize}

\end{frame}

%%%%%%%%%%%%%%%%%%%%%%%%%%%%%%%%%%%%%%%%%%%%%%%%%%%%%%%%%%%%%%%%%%%%%%%%%%%%%%%%

\begin{frame}

\newlibtitle

\bookpath{std/testing}:
A new Std/testing library, with standard utilities for building tests.
\begin{itemize}
\item
\code{misc/assert.lisp} and \code{misc/eval.lisp}
have been moved here and refactored into smaller files.
\item
This library contains utilities to build tests,
not actual tests (aside from perhaps some to test the testing utilities).
\end{itemize}

\separation

\bookpath{std/typed-alists}:
A new Std/typed-alists library, with alists of various types.
\begin{itemize}
\item
Analogous to Std/typed-lists.
\item
Add more typed alists, as needed.
\end{itemize}

\end{frame}

%%%%%%%%%%%%%%%%%%%%%%%%%%%%%%%%%%%%%%%%%%%%%%%%%%%%%%%%%%%%%%%%%%%%%%%%%%%%%%%%

\begin{frame}

\implibtitle

\bookpath{build}:
Books build system.
\begin{itemize}
\item \bookpath{build/*.certdep}: Track dependencies of books on certain ACL2 system features---see xdoc topic \code{acl2-system-feature-dependencies}.
\item Add \code{ifdef-(un)define(!)} for setting/unsetting environment
  variables (non-)locally, with effects on \code{if(n)def} forms
  understood by the build system.
\item \bookpath{build/cert.pl} supporting libraries refactored into proper (?) Perl modules.
\end{itemize}

\end{frame}

%%%%%%%%%%%%%%%%%%%%%%%%%%%%%%%%%%%%%%%%%%%%%%%%%%%%%%%%%%%%%%%%%%%%%%%%%%%%%%%%

\begin{frame}

\implibtitle

\bookpath{kestrel/apt}:
APT (Automated Program Transformations),
a toolkit to transform programs and program specifications
with automated support.
\begin{itemize}
\item
Added \code{isodata} and \code{propagate-iso},
the isomorphic data type transformations
described in the paper
\textit{Isomorphic Data Type Transformations} at this workshop.
\item
Added \code{parteval}, a partial evaluation transformation.
\item
Added \code{casesplit}, a case splitting transformation.
\item
Extended \code{tailrec}, the tail recursion transformation.
\item
Extended \code{restrict}, the domain restriction transformation.
\item
Added an APT defaults table
to customize certain behaviors of the APT transformations.
\item
Added some APT-specific XDOC constructors
for both user and developer documentation.
\end{itemize}

\end{frame}

%%%%%%%%%%%%%%%%%%%%%%%%%%%%%%%%%%%%%%%%%%%%%%%%%%%%%%%%%%%%%%%%%%%%%%%%%%%%%%%%

\begin{frame}

\implibtitle

\bookpath{kestrel/bitcoin}:
A library for the Bitcoin cryptocurrency and platform.
\begin{itemize}
\item
Formalized BIP (Bitcoin Improvement Proposal) 32, 39, 43, and 44.
(These are for cryptocurrency wallets.)
\item
Added verified executable attachments
for some declaratively specified functions.
\end{itemize}

\end{frame}

%%%%%%%%%%%%%%%%%%%%%%%%%%%%%%%%%%%%%%%%%%%%%%%%%%%%%%%%%%%%%%%%%%%%%%%%%%%%%%%%

\begin{frame}

\implibtitle

\bookpath{kestrel/ethereum}:
A library for the Ethereum cryptocurrency and platform.
\begin{itemize}
\item
Completed the RLP development, which started before ACL2-2018.
This development is described in the paper
\textit{Ethereum's Recursive Length Prefix in ACL2} at this workshop.
\item
Formalized Modified Merkle Patricia tree and the Ethereum database.
\item
Formalized the construction of signed transactions.
\item
Formalized the calculation of an account address from a public key.
\end{itemize}

\end{frame}

%%%%%%%%%%%%%%%%%%%%%%%%%%%%%%%%%%%%%%%%%%%%%%%%%%%%%%%%%%%%%%%%%%%%%%%%%%%%%%%%

\begin{frame}

\implibtitle

\bookpath{kestrel/fty}:
Extensions of the FTY library.
\begin{itemize}
\item
Added \code{deflist-of-len}, for lists of specified size.
\item
Added \code{defset}, for osets of specified types.
\item
Added \code{defomap}, for omaps of specified types.
\item
Improved \code{defbyte} and \code{defbytelist} to generate more theorems.
\item
Added \code{defbyte-ihs-theorems},
to generate theorems about \code{defbyte} fixtypes
and functions in the IHS library.
\item
Added \code{deffixequiv-sk},
to automate the proofs of \code{deffixequiv} for \code{defun-sk} functions.
\item
Added \code{defflatsum},
for flat (i.e.\ not tagged) sums of disjoint types.
\end{itemize}

\end{frame}

%%%%%%%%%%%%%%%%%%%%%%%%%%%%%%%%%%%%%%%%%%%%%%%%%%%%%%%%%%%%%%%%%%%%%%%%%%%%%%%%

\begin{frame}

\implibtitle

\bookpath{kestrel/hdwallet}:
A proof-of-concept hierarchical deterministic wallet for cryptocurrencies.
\begin{itemize}
\item
Uses components from the cryptographic, Bitcoin, and Ethereum libraries.
\item
Holds ether, but can be extended to other currencies.
\item
Packaged into a Docker image available on the Docker Hub.
Docker build information is included.
\item
See the file \code{kestrel/hdwallet/README.md}
for build and usage instructions.
\end{itemize}

\end{frame}

%%%%%%%%%%%%%%%%%%%%%%%%%%%%%%%%%%%%%%%%%%%%%%%%%%%%%%%%%%%%%%%%%%%%%%%%%%%%%%%%

\begin{frame}

\implibtitle

\bookpath{kestrel/java}:
A library for Java.
\begin{itemize}
\item
Significantly extended ATJ, the Java code generator for ACL2.
In particular, a shallow embedding approach is now supported.
Described in a rump talk at this workshop.
\item
Extended the Java language formalization with models of
various aspects of the language's syntax and semantics.
\item
Added a grammar of Java written in ABNF (Augmented Backus-Naur Form),
and processed it with the verified ABNF grammar parser in \code{kestrel/abnf}.
(This parses the ABNF grammar of Java, not Java.)
\end{itemize}

\end{frame}

%%%%%%%%%%%%%%%%%%%%%%%%%%%%%%%%%%%%%%%%%%%%%%%%%%%%%%%%%%%%%%%%%%%%%%%%%%%%%%%%

\begin{frame}

\implibtitle

\bookpath{kestrel/soft}:
Macros to mimic second-order functions and theorems.
\begin{itemize}
\item
Simplified macros to no longer require explicit function parameters.
\end{itemize}

\end{frame}

%%%%%%%%%%%%%%%%%%%%%%%%%%%%%%%%%%%%%%%%%%%%%%%%%%%%%%%%%%%%%%%%%%%%%%%%%%%%%%%%

\begin{frame}

\implibtitle

\bookpath{kestrel/utilities/digits-any-base}:
Conversions between natural numbers
and their representations in arbitrary bases.
\begin{itemize}
\item
Added functions to group and ungroup digits between larger and smaller bases.
\item
Added macros to generate specialized, more concise functions
for specified bases.
\end{itemize}

\separation

\bookpath{kestrel/utilities/strings}:
String utilities (to be eventually integrated with Std/strings).
\begin{itemize}
\item
Added functions to convert between strings or character lists
and even-length lists of hexadecimal digits.
\end{itemize}

\separation

\bookpath{kestrel/utilities/typed-lists}:
Typed list utilities (to be eventually integrated with Std/typed-lists).
\begin{itemize}
\item
Added \code{bit-listp}, with associated theorems.
\end{itemize}

\end{frame}

%%%%%%%%%%%%%%%%%%%%%%%%%%%%%%%%%%%%%%%%%%%%%%%%%%%%%%%%%%%%%%%%%%%%%%%%%%%%%%%%

\begin{frame}

\implibtitle

\bookpath{projects/filesystems}:
The filesystem books now cover many more system calls, include many
more cosimulation tests, and have verification of the transformations
between HiFAT, an abstract model of FAT32 with directory trees, and
LoFAT, the concrete model of FAT32 which replicates the on-disk FAT32
format. These verified transformations support refinement proofs
between LoFAT syscalls and their HiFAT equivalents, and these in turn
support proofs (of which examples can be seen in the directory)
to be carried out for real executable programs which use file
operations. Work is ongoing to make code proofs simpler and more
automatable with the use of separation logic. An earlier stage of
this work was covered in an ITP 2019 paper (Mehta, Cook.)

As by-products of this work, various improvements have been
contributed to the standard libraries, the Kestrel books and the ACL2
sources.

\end{frame}

%%%%%%%%%%%%%%%%%%%%%%%%%%%%%%%%%%%%%%%%%%%%%%%%%%%%%%%%%%%%%%%%%%%%%%%%%%%%%%%%

\begin{frame}

\implibtitle

\bookpath{projects/rac}:
Restricted Algorithmic C.
\begin{itemize}
\item
A new book, \code{books/projects/rac/lisp/internal-fns-gen.lisp},
which implements two tools that generate functions that compute
values of local (bound) variables of an input function
(to be used as described in Russinoff's workshop paper):
  \begin{itemize}
  \item \code{CONST-FNS-GEN} is applicable to
        functions with non-recursive definitions.
  \item \code{LOOP-FNS-GEN} can be applied to
        certain functions with restricted recursive definitions.
  \end{itemize}
\item
Modifications of \code{books/projects/rac/src}:
The RAC parser has been updated to check the placement restrictions
on return statements in RAC programs.
\end{itemize}

\end{frame}

%%%%%%%%%%%%%%%%%%%%%%%%%%%%%%%%%%%%%%%%%%%%%%%%%%%%%%%%%%%%%%%%%%%%%%%%%%%%%%%%

\begin{frame}

\implibtitle

\bookpath{projects/smtlink}:
Smtlink, a framework for integrating external SMT solvers into ACL2.
\begin{itemize}
\item
Allow Smtlink to handle functions using \code{cw}.
\item
Made it possible to use \code{:smtlink} hint inside an Smtlink proof.
\item
Added abstract type support for Smtlink.
\item
Fixed the problem of rewrite-loops by changing computed-hints structure in Smtlink.
\item
Fixed the {\color{red} \bf{soundness bug}} that \code{unary-/} can be interpreted as integer division in Smtlink.
\item
Added a link to an example using Smtlink for verifying an ASP* Pipeline. (This example will be moved to the ACL2 repository once we have the next stable version of Smtlink.)
\item
Made improvements to the documentation.
\end{itemize}

\end{frame}

%%%%%%%%%%%%%%%%%%%%%%%%%%%%%%%%%%%%%%%%%%%%%%%%%%%%%%%%%%%%%%%%%%%%%%%%%%%%%%%%

\begin{frame}

\implibtitle

\bookpath{projects/x86isa}:
X86ISA, the formal model of the x86 ISA.
\begin{itemize}
\item
Improved some aspects of the model of segmented memory.
\item
Added support for additional forms of the
\code{MOV}, \code{SHLD}, and \code{SHRD} instructions.
\item
Improved the \code{definst} macro to generate more boilerplate code.
\item
Updates to program instrumentation utilities to allow logging (parts of) the x86 state to a specified file in both CCL and SBCL.
\end{itemize}

\end{frame}

%%%%%%%%%%%%%%%%%%%%%%%%%%%%%%%%%%%%%%%%%%%%%%%%%%%%%%%%%%%%%%%%%%%%%%%%%%%%%%%%

\begin{frame}

\implibtitle

\bookpath{rtl}:
Register transfer logic library.
\begin{itemize}
\item
The book \code{books/rtl/rel11/lib/srt.lisp} now includes
a formalization of a radix-8 SRT square root algorithm.
\item
The book \code{books/rtl/rel11/lib/add.lisp} now includes
a correctness theorem for a generalized leading zero anticipator
that does not assume an ordering of its operands.
\end{itemize}

\end{frame}

%%%%%%%%%%%%%%%%%%%%%%%%%%%%%%%%%%%%%%%%%%%%%%%%%%%%%%%%%%%%%%%%%%%%%%%%%%%%%%%%

\begin{frame}

\implibtitle

\bookpath{std/alists}:
Standard association list library.
\begin{itemize}
\item
Added a function \code{remove-assocs},
which generalizes \code{remove-assoc} from single to multiple keys.
\item
Added functions \code{alist-map-keys} and \code{alist-map-vals},
which ignore shadowed pairs.
\item
Added some theorems.
\end{itemize}

\separation

\bookpath{std/basic}:
Standard basic library.
\begin{itemize}
\item
Added recognizers \code{bytep} and \code{nibblep}
for ``standard'' bytes and nibbles.
\item
Added \code{pos-fix}, a fixer for \code{posp}.
\end{itemize}

\end{frame}

%%%%%%%%%%%%%%%%%%%%%%%%%%%%%%%%%%%%%%%%%%%%%%%%%%%%%%%%%%%%%%%%%%%%%%%%%%%%%%%%

\begin{frame}

\implibtitle

\bookpath{std/lists}:
Standard list library.
\begin{itemize}
\item
The function \code{list-fix} has been made built-in,
with the name \code{true-list-fix}.
\item
The \code{take-redefinition} theorem has been removed,
because the built-in definition of \code{take} has been
changed to be like that redefinition.
\item
Added a file with theorems about \code{union-equal}.
\end{itemize}

\separation

\bookpath{std/strings}:
Standard string library.
\begin{itemize}
\item
Added a variant \code{strtok!} of \code{strtok}.
It does not treat multiple contiguous delimiters like one.
\item
Added \code{printtree} library for efficiently composing large strings from pieces.
\end{itemize}

\end{frame}

%%%%%%%%%%%%%%%%%%%%%%%%%%%%%%%%%%%%%%%%%%%%%%%%%%%%%%%%%%%%%%%%%%%%%%%%%%%%%%%%

\begin{frame}

\implibtitle

\bookpath{std/typed-lists}:
Standard typed lists library.
\begin{itemize}
\item
Added lists of strings and symbols.
(Originally in \code{system/pseudo-good-worldp.lisp}.)
\end{itemize}

\separation

\bookpath{std/util}:
Standard utility library.
\begin{itemize}
\item
The
\code{defthm-natp},
\code{defthm-unsigned-byte-p}, and
\code{defthm-signed-byte-p}
utilities have been moved here from the X86ISA model.
\item
The \code{use-termhint} utility has been moved here
from \code{clause-processors/use-hint.lisp}.
\item Added \code{defret-mutual-generate}, discussed in the paper \textit{Generating Mutually Inductive Theorems from Concise Descriptions} at this workshop.
\item
Like \code{defun-nx}, \code{define} macro also disables the executable-counterpart of a non-executable function now.
\end{itemize}

\end{frame}

%%%%%%%%%%%%%%%%%%%%%%%%%%%%%%%%%%%%%%%%%%%%%%%%%%%%%%%%%%%%%%%%%%%%%%%%%%%%%%%%

\begin{frame}

\implibtitle

\bookpath{system/pseudo-good-worldp.lisp}:
Factored out some reusable functions.
\begin{itemize}
\item
Predicates for
event forms,
command forms,
event landmarks,
command landmarks, and
tests-and-calls structures
are now in separate files.
\item
More could be factored out.
\item
These could be perhaps integrated with Std/system.
\end{itemize}

\end{frame}

%%%%%%%%%%%%%%%%%%%%%%%%%%%%%%%%%%%%%%%%%%%%%%%%%%%%%%%%%%%%%%%%%%%%%%%%%%%%%%%%

\begin{frame}

\implibtitle

\bookpath{xdoc/constructors.lisp}:
Made some improvements and additions.

\separation

\bookpath{xdoc/defxdoc-plus.lisp}:
Added \code{defxdoc+}, an enhancement of \code{defxdoc}
that supports more concise expression of
ordering of subtopics and default parents.

\end{frame}

%%%%%%%%%%%%%%%%%%%%%%%%%%%%%%%%%%%%%%%%%%%%%%%%%%%%%%%%%%%%%%%%%%%%%%%%%%%%%%%%

\end{document}
