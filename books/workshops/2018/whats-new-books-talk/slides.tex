\documentclass{beamer}
\usecolortheme{seahorse}
\usepackage{inconsolata}
\usepackage{alltt}

\beamertemplatenavigationsymbolsempty

\setbeamertemplate{footline}[frame number]

%%%%%%%%%%%%%%%%%%%%%%%%%%%%%%%%%%%%%%%%%%%%%%%%%%%%%%%%%%%%%%%%%%%%%%%%%%%%%%%%

\newcommand{\code}[1]{\texttt{#1}}
\newenvironment{codeblock}{\begin{alltt}}{\end{alltt}}

\newcommand{\bookpath}[1]{\textbf{\code{#1}}}

\newcommand{\newlibtitle}{\frametitle{New Libraries}}
\newcommand{\implibtitle}{\frametitle{Improved Libraries}}

%%%%%%%%%%%%%%%%%%%%%%%%%%%%%%%%%%%%%%%%%%%%%%%%%%%%%%%%%%%%%%%%%%%%%%%%%%%%%%%%

\begin{document}

%%%%%%%%%%%%%%%%%%%%%%%%%%%%%%%%%%%%%%%%%%%%%%%%%%%%%%%%%%%%%%%%%%%%%%%%%%%%%%%%

\title{What's New in the Community Books}

\subtitle{Since the ACL2-2017 Workshop}

\author{Cuong Chau\inst{9},
        Alessandro Coglio\inst{5},
        Jared Davis\inst{1},
        \\
        Andrew Gacek\inst{7},
        Shilpi Goel\inst{3},
        Mark Greenstreet\inst{8},
        \\
        David Greve\inst{7},
        Matt Kaufmann\inst{9},
        Keshav Kini\inst{6},
        Carl Kwan\inst{8},
        \\
        Mihir Mehta\inst{9},
        J Moore\inst{9},
        David Russinoff\inst{2},
        Julien Schmaltz\inst{4},
        \\
        Rob Sumners\inst{3},
        Sol Swords\inst{3}}

\institute{\inst{1}Apple,
           \inst{2}ARM,
           \inst{3}Centaur,
           \inst{4}Eindhoven Institute of Technology,
           \inst{5}Kestrel Institute,
           \inst{6}Oracle,
           \inst{7}Rockwell Collins,
           \inst{8}University of British Columbia, \\
           \inst{9}University of Texas at Austin}

\date{ACL2-2018 Workshop}

\frame[noframenumbering,plain]{\titlepage}

%%%%%%%%%%%%%%%%%%%%%%%%%%%%%%%%%%%%%%%%%%%%%%%%%%%%%%%%%%%%%%%%%%%%%%%%%%%%%%%%

% GENERAL FORMAT OF EACH SLIDE:

%\begin{frame}

%\newlibtitle or \implibtitle % for new vs. improved libraries

%\bookpath{path/to/the/library}: % directory or file, use macro for font
% Short description of what this library is, 1-2 lines.
%\begin{itemize}
%\item
% A highlight about this new or improved library.
%\item
% Another highlight.
%\item
% Say if it's described in a paper at this workshop,
% in which case, there's little to say here at all.
%\end{itemize}

%\ % to separate the next entry, if any

%\bookpath{another/path}: % as above

%\end{frame}

%%%%%%%%%%%%%%%%%%%%%%%%%%%%%%%%%%%%%%%%%%%%%%%%%%%%%%%%%%%%%%%%%%%%%%%%%%%%%%%%

% ORDER OF THE SLIDES AND ENTRIES:

% First all the new libraries, in alphabetical order.

% Then all the improve libraries, in alphabetical order.

% Then one or more slides with additional contributions.

% Don't worry about adding your contributions in order:
% Alessandro will quickly put them in order if needed,
% and also fix any formatting issues.

%%%%%%%%%%%%%%%%%%%%%%%%%%%%%%%%%%%%%%%%%%%%%%%%%%%%%%%%%%%%%%%%%%%%%%%%%%%%%%%%

% OTHER REMARKS:

% If you want to show code block snippets (as done for kestrel/utilities/xdoc),
% add [fragile] just after \begin{frame}.

% For guidelines on content and level of detail,
% please see existing entries.
% The goal is to provide enough information to get possible users interested.

%%%%%%%%%%%%%%%%%%%%%%%%%%%%%%%%%%%%%%%%%%%%%%%%%%%%%%%%%%%%%%%%%%%%%%%%%%%%%%%%

\begin{frame}

\newlibtitle

\bookpath{build/ifdef.lisp}:
Defines \code{ifdef} and \code{ifndef} forms which test environment variables;
supported by the build system.

\

\bookpath{centaur/acre}:
New regular expression implementation supporting features somewhat similar to Perl regexes.

\

\bookpath{centaur/bitops/sparseint.lisp}:
Library representing bignums as balanced trees to efficiently support
operations that preserve large ranges of bits.

\

\bookpath{centaur/glmc}:
Interface to hardware model checkers.

\

\bookpath{centaur/truth}:
Integer-encoded truth table library.

\end{frame}

%%%%%%%%%%%%%%%%%%%%%%%%%%%%%%%%%%%%%%%%%%%%%%%%%%%%%%%%%%%%%%%%%%%%%%%%%%%%%%%%

\begin{frame}

\newlibtitle

\bookpath{kestrel/apt}:
APT (Automated Program Transformations),
a toolkit
to transform programs and program specifications with automated support.
\begin{itemize}
\item
Includes two of Kestrel's $\sim$40 transformations.
\item
Also includes some utilities used across transformations.
\item
More forthcoming.
\end{itemize}

\

\bookpath{kestrel/auto-termination}:
\code{defunt} is a variant of \code{defun} that can prove termination
using previously-proved termination theorems from a large set of
community books, as described in the paper
\textit{DefunT: A Tool for Automating Termination Proofs
by Using the Community Books}
at this workshop.

\

\bookpath{kestrel/bitcoin}:
A (small start towards a) library for the Bitcoin cryptocurrency and ecosystem.
\begin{itemize}
\item
Executable specification of Base58 encoding and decoding.
\item
Executable specification of Base58Check encoding.
\end{itemize}

\end{frame}

%%%%%%%%%%%%%%%%%%%%%%%%%%%%%%%%%%%%%%%%%%%%%%%%%%%%%%%%%%%%%%%%%%%%%%%%%%%%%%%%

\begin{frame}

\newlibtitle

\bookpath{kestrel/ethereum}:
A library for the Ethereum
cryptocurrency and ecosystem.
\begin{itemize}
\item
Executable specification of RLP (Recursive Length Prefix) encoding;
declarative specification of RLP decoding.
\item
Executable specification of hex-prefix encoding.
\item
Kestrel is actively working on this.
\end{itemize}

\

\bookpath{kestrel/java}:
A library for Java.
\begin{itemize}
\item
AIJ (ACL2 In Java), the deep embedding of ACL2 in Java
described in the paper
\textit{A Simple Java Code Generator for ACL2
Based on a Deep Embedding of ACL2 in Java}
at this Workshop.
\item
ATJ (ACL2 To Java), the Java code generator for ACL2
described in the paper just above.
\end{itemize}

\end{frame}

%%%%%%%%%%%%%%%%%%%%%%%%%%%%%%%%%%%%%%%%%%%%%%%%%%%%%%%%%%%%%%%%%%%%%%%%%%%%%%%%

\begin{frame}

\newlibtitle

\bookpath{kestrel/utilities/apply-fn-if-known.lisp}:
Apply a function, expressed as a package and a name, if it exists.

\

\bookpath{kestrel/utilities/auto-instance.lisp}:
\code{defthm<w} will attempt to prove a theorem directly
from previously-proved theorems by generating
suitable hints, using \code{previous-subsumer-hints}.

\

\bookpath{kestrel/utilities/digits-any-base}:
Conversions between natural numbers
and their representations in arbitrary bases.
\begin{itemize}
\item
Big and little endian.
\item
Minimal, minimal non-zero, or specified length.
\item
Several theorems, e.g.\ about inversions.
\end{itemize}

\

\bookpath{kestrel/utilities/er-soft-plus.lisp}:
The logic-mode utilities \code{er-soft+} and \code{er-soft-logic} produce
soft errors with specified error triples.

\end{frame}

%%%%%%%%%%%%%%%%%%%%%%%%%%%%%%%%%%%%%%%%%%%%%%%%%%%%%%%%%%%%%%%%%%%%%%%%%%%%%%%%

\begin{frame}

\newlibtitle

\bookpath{kestrel/utilities/fixbytes}:
Fixtypes for unsigned and signed bytes, and true lists thereof.
\begin{itemize}
\item
Macros to create fixtypes and theorems for a specified size.
The size may be a constrained nullary function,
e.g.\ useful to formalize C bytes.
\item
Several instances available; just include the respective file(s).
\item
These are candidate extensions of the \code{fty} library.
\end{itemize}

\

\bookpath{kestrel/utilities/include-book-paths.lisp}:
List paths via \code{include-book} down to a given book; may be useful
for reducing book dependencies.

\end{frame}

%%%%%%%%%%%%%%%%%%%%%%%%%%%%%%%%%%%%%%%%%%%%%%%%%%%%%%%%%%%%%%%%%%%%%%%%%%%%%%%%

\begin{frame}

\newlibtitle

\bookpath{kestrel/utilities/integer-range-*.lisp}:
Utilities related to \code{integer-range-p}.
\begin{itemize}
\item
Parameterized recognizer \code{integer-range-listp}.
\item
Parameterized fixers
\code{integer-range-fix} and \code{integer-range-list-fix}.
\item
Several theorems.
\end{itemize}

\

\bookpath{kestrel/utilities/magic-macroexpand.lisp}:
Logic-mode macroexpansion.

\

\bookpath{kestrel/utilities/messages.lisp}:
A few utilities for \code{msgp} values,
e.g.\ to convert the first character to upper/lower case.

\

\bookpath{kestrel/utilities/orelse.lisp}:
Try one event, then a second one if the first fails.

\end{frame}

%%%%%%%%%%%%%%%%%%%%%%%%%%%%%%%%%%%%%%%%%%%%%%%%%%%%%%%%%%%%%%%%%%%%%%%%%%%%%%%%

\begin{frame}

\newlibtitle

\bookpath{kestrel/utilities/proof-builder-macros.lisp}:
A book that defines some proof-builder macros.  Current contents
include definitions of:
\begin{itemize}
\item
\code{when-not-proved} to skip instructions when all goals have
been proved;
\item
\code{prove-guard} and \code{prove-termination}, for using
previously-proved guard or termination theorems efficiently; and
\item
a more general macro, \code{fancy-use}, for using lemma
instances efficiently.
\end{itemize}

\

\bookpath{kestrel/utilities/skip-in-book.lisp}:
The utility, \code{skip-in-book}, wraps around a form to prevent its
evaluation during book certification or inclusion.

\

\bookpath{kestrel/utilities/symbols.lisp}:
Some utilities for symbols.
\begin{itemize}
\item
These could become a new \code{std/symbols} library.
\end{itemize}

\end{frame}

%%%%%%%%%%%%%%%%%%%%%%%%%%%%%%%%%%%%%%%%%%%%%%%%%%%%%%%%%%%%%%%%%%%%%%%%%%%%%%%%

\begin{frame}[fragile]

\newlibtitle

\bookpath{kestrel/utilities/system/paired-names.lisp}:
Utilities for names consisting of
two parts with a customizable separator in between.
(Used by APT, but more general.)

\

\bookpath{kestrel/utilities/untranslate-preprocessing.lisp}:
A macro \code{add-const-to-untranslate-preprocess}
to keep a named constant unexpanded in the screen output.

\

\bookpath{kestrel/utilities/xdoc}:
XDOCumentation utilities.
\begin{itemize}
\item
Constructors of well-tagged XDOC strings, e.g.
\begin{codeblock}
(xdoc::p "This is a paragraph.")
(xdoc::ul
  (xdoc::li "First unordered item.")
  (xdoc::li "Second unordered item."))
\end{codeblock}
\item
\code{defxdoc+} extends \code{defxdoc} with
\code{:order-subtopics t/nil} and \code{:default-parent t/nil}.
\item
These are candidate extensions of the \code{xdoc} library.
\end{itemize}

\end{frame}

%%%%%%%%%%%%%%%%%%%%%%%%%%%%%%%%%%%%%%%%%%%%%%%%%%%%%%%%%%%%%%%%%%%%%%%%%%%%%%%%

\begin{frame}

\newlibtitle

\bookpath{projects/async/tools/convert-edif.lisp}:
Convert between EDIF format and a convenient s-expression format.

\

\bookpath{projects/sat/zz-resolution-checker}:
An early SAT proof-checker from 2011 based on resolution (see README).

\

\bookpath{std/stobjs/updater-independence}:
Utility for defining stobj and stobj-like accessor/updater
independence theorems.

\

\bookpath{std/util/termhints}:
Hint utility described in the paper
\textit{Hint Orchestration Using ACL2's Simplifier}
at this Workshop.

\

\bookpath{tools/run-script.lisp}:
This utility supports testing of evaluation of the forms
in a given file, to check that the output is as expected.
Several community books utilize it.

\end{frame}

%%%%%%%%%%%%%%%%%%%%%%%%%%%%%%%%%%%%%%%%%%%%%%%%%%%%%%%%%%%%%%%%%%%%%%%%%%%%%%%%

\begin{frame}

\newlibtitle

\bookpath{workshops/2018/gamboa-cowles}:
Supporting materials for the paper
\textit{The Fundamental Theorem of Algebra in ACL2}
at this Workshop.

\

\bookpath{workshops/2018/greve-gacek}:
Supporting materials for the paper
\textit{Trapezoidal Generalization over Linear Constraints}
at this Workshop.

\

\bookpath{workshops/2018/kaufmann}:
Supporting materials for the paper
\textit{DefunT: A Tool for Automating Termination Proofs
by Using the Community Books}
at this Workshop.
(This is an archived version of
\code{defunt} in \code{kestrel/auto-termination}).

\end{frame}

%%%%%%%%%%%%%%%%%%%%%%%%%%%%%%%%%%%%%%%%%%%%%%%%%%%%%%%%%%%%%%%%%%%%%%%%%%%%%%%%

\begin{frame}

\newlibtitle

\bookpath{workshops/2018/kwan-greenstreet}:
Supporting materials for the papers
\textit{The Cauchy-Schwarz Inequality in ACL2(r)} and
\textit{Convex Functions in ACL2(r)}
at this Workshop.

\

\bookpath{workshops/2018/mehta}:
Supporting materials for the paper
\textit{Formalising Filesystems in the ACL2 Theorem Prover:
an Application to FAT32}
at this Workshop.
(But see also \code{projects/filesystems}.)

\

\bookpath{workshops/2018/sumners}:
Supporting materials for the paper
\textit{A Toolbox for Property Checking from Simulation Using Incremental SAT}
at this Workshop.

\

\

\

The supporting material for other papers at this Workshop
are elsewhere, not under \code{workshops/2018}.

\end{frame}

%%%%%%%%%%%%%%%%%%%%%%%%%%%%%%%%%%%%%%%%%%%%%%%%%%%%%%%%%%%%%%%%%%%%%%%%%%%%%%%%

\begin{frame}

\newlibtitle

TODO: add more contributions

\end{frame}

%%%%%%%%%%%%%%%%%%%%%%%%%%%%%%%%%%%%%%%%%%%%%%%%%%%%%%%%%%%%%%%%%%%%%%%%%%%%%%%%

\begin{frame}

\implibtitle

\bookpath{centaur/aignet}:
And-Inverter Graph (AIG) representation
for Boolean functions and finite-state machines.
\begin{itemize}
\item
New verified AIGNET transforms including FRAIGing, DAG-aware balancing
and rewriting.
\item
AIGNET natively supports XORs, i.e.\ represents them using one node
instead of three.
\end{itemize}

\

\bookpath{centaur/bitops/rotate.lisp}:
Bit-vector rotation libraries.
\begin{itemize}
\item
Generalized existing theorems and added a new theorem for compositions
of \code{rotate-left} operations, as well as a theorem for
compositions of \code{rotate-right} operations.
\item
To do: Add theorems for compositions of \code{rotate-left} and
\code{rotate-right} with each other.
\end{itemize}

\end{frame}

%%%%%%%%%%%%%%%%%%%%%%%%%%%%%%%%%%%%%%%%%%%%%%%%%%%%%%%%%%%%%%%%%%%%%%%%%%%%%%%%

\begin{frame}

\implibtitle

\bookpath{centaur/fty/bitstruct}:
Define a bit vector type with accessor/updater functions for its fields.
\begin{itemize}
\item
The \code{:exec} part of the \code{mbe} in accessor and updater
functions now has efficient, heavily type-declared code that avoids
bignum operations whenever possible.
\item
Accessor and updater functions can now be inlined.
\end{itemize}

\

\bookpath{centaur/gl}:
Symbolic simulation framework for solving finite theorems.
\begin{itemize}
\item
Add hooks in GL to allow calling AIGNET transforms before SAT.
\item
Improve GL counterexample generation for term-level reasoning.
\item
Added accumulated-persistence-like rule profiling.
\end{itemize}

\end{frame}

%%%%%%%%%%%%%%%%%%%%%%%%%%%%%%%%%%%%%%%%%%%%%%%%%%%%%%%%%%%%%%%%%%%%%%%%%%%%%%%%

\begin{frame}

\implibtitle

\bookpath{centaur/sv}:
Hardware verification library with vector-based expression representation.
\begin{itemize}
\item
Many SV/SVEX algorithms are now based on \code{sparseints} so that they scale
when dealing with variables thousands/millions of bits in size.
\end{itemize}

\

\bookpath{centaur/vl}:
Library for SystemVerilog and regular Verilog.
\begin{itemize}
\item
Add new SystemVerilog lint check based on accurately determining
used/set ranges of vectors.
\end{itemize}

\

\bookpath{coi/...}:
TODO: add

\end{frame}

%%%%%%%%%%%%%%%%%%%%%%%%%%%%%%%%%%%%%%%%%%%%%%%%%%%%%%%%%%%%%%%%%%%%%%%%%%%%%%%%

\begin{frame}

\implibtitle

\bookpath{kestrel/soft}:
SOFT (Second-Order Functions and Theorems),
macros to mimic second-order functions and theorems.
\begin{itemize}
\item
Added full support for \code{defun-sk2}.
\item
Improved user interface.
\end{itemize}

\

\bookpath{kestrel/utilities/...}:
Started refactoring some of these utilities to reduce book dependencies.

\

\bookpath{kestrel/utilities/copy-def.lisp}:
Made improvements: better handling of \code{mutual-recursion} and of
the \code{:equiv} argument, and generated \code{:expand} hint for better
handling of recursion.

\

\bookpath{kestrel/utilities/directed-untranslate.lisp}:
Made several improvements to \code{directed-untranslate}, in particular
for \code{let}, \code{let*}, \code{mv}, \code{mv-let}, and \code{b*},
including enhanced executability of the result.

\end{frame}

%%%%%%%%%%%%%%%%%%%%%%%%%%%%%%%%%%%%%%%%%%%%%%%%%%%%%%%%%%%%%%%%%%%%%%%%%%%%%%%%

\begin{frame}

\implibtitle

\bookpath{kestrel/utilities/error-checking.lisp}:
Utilities to check error conditions and return customizable error messages.
\begin{itemize}
\item
Improved the \code{def-error-checker} macro,
e.g.\ to support logic-mode error-checking functions.
\item
Added several error-checking functions.
\end{itemize}

\

\bookpath{kestrel/utilities/osets.lisp}:
Utilities for osets.
\begin{itemize}
\item
Added a fixtype for osets.
\item
These are candidate extensions of the \code{std/osets} library.
\end{itemize}

\

\bookpath{kestrel/utilities/strings}:
String manipulation libraries.
\begin{itemize}
\item Added several new rewrite rules.
\end{itemize}

\end{frame}

%%%%%%%%%%%%%%%%%%%%%%%%%%%%%%%%%%%%%%%%%%%%%%%%%%%%%%%%%%%%%%%%%%%%%%%%%%%%%%%%

\begin{frame}

\implibtitle

\bookpath{kestrel/utilities/user-interface.lisp}:
Utilities for customizing screen output of user-defined events.
\begin{itemize}
\item
Added several utilities.
\end{itemize}

\

\bookpath{kestrel/utilities/system/defun-sk-queries.lisp}:
Utilities to query \code{defun-sk} functions.
\begin{itemize}
\item
Added support for the recently added \code{:constrain} option.
\item
These could become part of a new \code{std/system} library.
\end{itemize}

\

\bookpath{kestrel/utilities/system/terms.lisp}:
Utilities to manipulate terms.
\begin{itemize}
\item
Added and improved several utilities.
\item
Moved some utilities to a separate file
\code{term-function-recognizers.lisp}.
\item
These could become part of a new \code{std/system} library.
\end{itemize}

\end{frame}

%%%%%%%%%%%%%%%%%%%%%%%%%%%%%%%%%%%%%%%%%%%%%%%%%%%%%%%%%%%%%%%%%%%%%%%%%%%%%%%%

\begin{frame}

\implibtitle

\bookpath{kestrel/utilities/system/world-queries.lisp}:
Utilities to query worlds.
\begin{itemize}
\item
Added and improved several utilities.
\item
There are two variants for most of these utilities:
a ``fast'' one and a ``logic-friendly'' one
(see documentation for details).
\item
These could become part of a new \code{std/system} library.
\end{itemize}

\

\bookpath{misc/assert.lisp} \& \bookpath{misc/eval.lisp}:
Testing utilities.
\begin{itemize}
\item
Added some utilities moved from \code{kestrel/utilities/testing.lisp}.
\item
Added some XDOCumentation.
\item
Renamed some utilities for greater uniformity
(deprecated the old names).
\item
Reduced book dependencies.
\end{itemize}

\end{frame}

%%%%%%%%%%%%%%%%%%%%%%%%%%%%%%%%%%%%%%%%%%%%%%%%%%%%%%%%%%%%%%%%%%%%%%%%%%%%%%%%

\begin{frame}

\implibtitle

\bookpath{misc/expander.lisp}:
The expander has been improved in several ways.

\

\bookpath{misc/install-not-normalized.lisp}:
Improved \code{install-not-normalized} to handle cases in which
recursively-defined functions have non-recursive normalized
definitions.

\

\bookpath{misc/profiling.lisp}:
Profiling fixes for recent distributions of CCL.

\

\bookpath{projects/apply} \& \bookpath{projects/apply-model}:
Updated books pertaining to \code{apply\$}.

\end{frame}

%%%%%%%%%%%%%%%%%%%%%%%%%%%%%%%%%%%%%%%%%%%%%%%%%%%%%%%%%%%%%%%%%%%%%%%%%%%%%%%%

\begin{frame}

\implibtitle

\bookpath{projects/async}:
ASYNC, the framework for modeling and verifying the functional correctness
of asynchronous (self-timed) circuit models.
\begin{itemize}
\item
Developed a new compositional methodology for scalable formal
verification of functional properties of self-timed circuit designs.
\item
Verified the functional correctness of data-loop-free self-timed
circuits (see \code{fifo/}).
\item
Verified the functional correctness of a self-timed serial
adder/subtractor model (see \code{serial-adder/}).
\item
Verified the functional correctness of iterative self-timed circuit
models that compute the greatest-common-divisor (GCD) (see \code{gcd/}).
\item
Verified the functional correctness of self-timed circuits performing
arbitrated merge operations (see \code{arbitration/}).
\end{itemize}

\end{frame}

%%%%%%%%%%%%%%%%%%%%%%%%%%%%%%%%%%%%%%%%%%%%%%%%%%%%%%%%%%%%%%%%%%%%%%%%%%%%%%%%

\begin{frame}

\implibtitle

\bookpath{projects/filesystems}:
Formal models of filesystems.
\begin{itemize}
\item
\texttt{M1} and \texttt{M2}, new filesystem models for FAT32,
described in the paper
\textit{Formalising Filesystems in the ACL2 Theorem Prover:
an Application to FAT32}
at this Workshop.
\end{itemize}

\

\bookpath{projects/sat/lrat}:
SAT proof-checker extensions (improved theorem, extension to
cube-and-conquer; see README).

\

\bookpath{projects/smtlink}:
Smtlink, a framework for integrating external SMT solvers into ACL2.
\begin{itemize}
\item
Smtlink has experienced great architecture refactoring and was moved from
\code{workshop/2015/peng-greenstreet} to \code{projects/smtlink},
as described in the paper \textit{Smtlink 2.0} at this Workshop.
\item
Developed new XDOC documentation.
\item
Added more toy examples and a ring oscillator proof example.
\end{itemize}

\end{frame}

%%%%%%%%%%%%%%%%%%%%%%%%%%%%%%%%%%%%%%%%%%%%%%%%%%%%%%%%%%%%%%%%%%%%%%%%%%%%%%%%

\begin{frame}

\implibtitle

{\small
\bookpath{projects/x86isa}:
X86ISA, the formal model of the x86 ISA.
\begin{itemize}
\item
Added support for 32-bit mode; see the paper
\textit{Adding 32-bit Mode to the ACL2 Model of the x86 ISA}
at this Workshop.
\item
Improved and extended some documentation.
\item
The model's modes are now called ``views'' to avoid overloading the
word ``mode'', which refers to an x86 processor's own modes of
operation.
\item
Opcode dispatch functions and coverage data are generated from
annotated opcode maps, which are taken from the Intel manuals.
\item
Added support for decoding VEX- and EVEX-encoded instructions
(AVX/AVX2/AVX512).
\item
Decode-time exceptions are detected during opcode dispatch now, as
opposed to inside individual instruction semantic functions.
\item
Added support for enabling/disabling machine features that depend on
CPUID feature flags.
\item
Codewalker can now be used to reason about x86 programs.
\end{itemize}
}

\end{frame}

%%%%%%%%%%%%%%%%%%%%%%%%%%%%%%%%%%%%%%%%%%%%%%%%%%%%%%%%%%%%%%%%%%%%%%%%%%%%%%%%

\begin{frame}

\implibtitle

\bookpath{std/io/combine.lisp}:
Byte-combining libraries.
\begin{itemize}
\item
Added invertibility theorems for \code{combine16u} and \code{combine32u}.
\item
To do: make these invertibility theorems compatible with \code{part-select}.
\item
To do: prove similar theorems for \code{combine64u} as well as for the
signed-integer functions, \code{combine16s} et al.
\end{itemize}

\end{frame}

%%%%%%%%%%%%%%%%%%%%%%%%%%%%%%%%%%%%%%%%%%%%%%%%%%%%%%%%%%%%%%%%%%%%%%%%%%%%%%%%

\begin{frame}

\implibtitle

\bookpath{tools/flag.lisp}:
The new keyword argument \code{:last-body} of \code{make-flag}
specifies use of the most recent definition rule.

\

\bookpath{tools/include-raw.lisp}:
Fixed an issue with option \code{:do-not-compile t} by extending
``fns-with-raw-code'' state globals.

\

\bookpath{tools/removable-runes.lisp}:
Improved \code{removable-runes} and added related utility,
\code{minimal-runes}, which returns a list of runes to enable that is
sufficient for admitting a given event.

\end{frame}

%%%%%%%%%%%%%%%%%%%%%%%%%%%%%%%%%%%%%%%%%%%%%%%%%%%%%%%%%%%%%%%%%%%%%%%%%%%%%%%%

\begin{frame}

\implibtitle

TODO: add more contributions

\end{frame}

%%%%%%%%%%%%%%%%%%%%%%%%%%%%%%%%%%%%%%%%%%%%%%%%%%%%%%%%%%%%%%%%%%%%%%%%%%%%%%%%

\begin{frame}

\frametitle{Additional Contributions}

\bookpath{workshops/references}:
BibTeX references for all the ACL2 Workshop papers,
and a LaTeX document that shows them.
TODO: add/improve as needed

\

\bookpath{xdoc/fancy/lib/katex}:
KaTeX, a JavaScript library for TeX math rendering on the web,
has been updated to version 0.8.3.

\

\textbf{Developers Guide}:
The topic \code{developers-guide} is, together with its subtopics, actually a
manual for ACL2 development.  It is intended for experienced ACL2
users who may wish to become ACL2 developers.

\end{frame}

%%%%%%%%%%%%%%%%%%%%%%%%%%%%%%%%%%%%%%%%%%%%%%%%%%%%%%%%%%%%%%%%%%%%%%%%%%%%%%%%

\end{document}
