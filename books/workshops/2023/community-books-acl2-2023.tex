% GENERAL FORMAT OF EACH SLIDE
% ----------------------------

%\begin{frame}

%\newlibtitle or \implibtitle % for new vs. improved libraries

%\bookpath{path/to/the/library}: % directory or file, use macro for font
% Short description of what this library is, 1-2 lines.
%\begin{itemize}
%\item
% A highlight about this new or improved library.
%\item
% Another highlight.
%\item
% Say if it's described in a paper at this workshop,
% in which case, there may be little to say here.
%\end{itemize}

%\separation % to separate the next entry, if any

%\bookpath{another/path}: % as above

%\end{frame}

% ORDER OF THE SLIDES AND ENTRIES
% -------------------------------

% First all the new libraries, in alphabetical order.

% Then all the improved libraries, in alphabetical order.

% Then one or more slides with additional contributions.

% Don't worry about adding your contributions in order:
% Alessandro will quickly put them in order if needed,
% and also fix any formatting issues.

% TITLE SLIDE
% -----------

% Add your name as co-author, in alphabetical order of last name.

% Add your organization, in alphabetical order.
% Renumber the \inst{n} instances as needed
% (both inside \author{...} and inside \institute{...}).

% Or ask Alessandro to do this.

% OTHER REMARKS
% -------------

% Use the \code{...} macro for code (i.e.\ fixed-width) font within the text.

% If you want to show code block snippets,
% use the verbatim environment,
% and add [fragile] just after \begin{frame}.

% For guidelines on content and level of detail,
% please see existing entries.
% The goal is to provide enough information to get possible users interested.

%%%%%%%%%%%%%%%%%%%%%%%%%%%%%%%%%%%%%%%%%%%%%%%%%%%%%%%%%%%%%%%%%%%%%%%%%%%%%%%%

\documentclass{beamer}
\usetheme{Copenhagen}
\usecolortheme{whale}
\usepackage{inconsolata}

\beamertemplatenavigationsymbolsempty

\setbeamertemplate{footline}[frame number]

%%%%%%%%%%%%%%%%%%%%%%%%%%%%%%%%%%%%%%%%%%%%%%%%%%%%%%%%%%%%%%%%%%%%%%%%%%%%%%%%

\newcommand{\code}[1]{\texttt{#1}}
\newenvironment{codeblock}{\begin{alltt}}{\end{alltt}}

\newcommand{\bookpath}[1]{\textbf{\code{#1}}}

\newcommand{\newlibtitle}{\frametitle{New Libraries}}
\newcommand{\implibtitle}{\frametitle{Improved Libraries}}

\newcommand{\separation}{\vspace*{1.5ex}}

%%%%%%%%%%%%%%%%%%%%%%%%%%%%%%%%%%%%%%%%%%%%%%%%%%%%%%%%%%%%%%%%%%%%%%%%%%%%%%%%

\title{What's New in the Community Books}

\subtitle{Since the ACL2-2022 Workshop}

\author{Alessandro Coglio\inst{1,2,3} (presenter),
        \emph{more authors}\inst{4}}

\institute{\inst{1}Aleo Systems Inc.,
           \inst{2}Kestrel Institute,
           \inst{3}Kestrel Technology,
           \inst{4}\emph{more organizations}}

\date{ACL2-2023 Workshop}

%%%%%%%%%%%%%%%%%%%%%%%%%%%%%%%%%%%%%%%%%%%%%%%%%%%%%%%%%%%%%%%%%%%%%%%%%%%%%%%%

\begin{document}

%%%%%%%%%%%%%%%%%%%%%%%%%%%%%%%%%%%%%%%%%%%%%%%%%%%%%%%%%%%%%%%%%%%%%%%%%%%%%%%%

\frame[noframenumbering,plain]{\titlepage}

%%%%%%%%%%%%%%%%%%%%%%%%%%%%%%%%%%%%%%%%%%%%%%%%%%%%%%%%%%%%%%%%%%%%%%%%%%%%%%%%

\begin{frame}

\frametitle{Overview}
      
\begin{itemize}
\item Almost 6,000 non-merge commits since the last Workshop.
\item From several contributors from several organizations.
\item Spanning hardware, mathematics, cryptography, blockchain,
      programming languages, virtual machines, machine code,
      standards, analysis, synthesis, and more.
\item These slides provides
      a more succinct overview than the book release notes,
      ordered by book path within each of the new and improved library parts.
\end{itemize}
      
\end{frame}

%%%%%%%%%%%%%%%%%%%%%%%%%%%%%%%%%%%%%%%%%%%%%%%%%%%%%%%%%%%%%%%%%%%%%%%%%%%%%%%%

\begin{frame}

\implibtitle

\bookpath{kestrel/abnf}:
Augmented Backus-Naur Form (ABNF).
\begin{itemize}
\item Refactored to organize constituents more clearly.
\item Significant extensions and improvements to the parsing generation tools.
\item New tools for ingesting grammars
      and generating operations and theorems for them.
\end{itemize}

\end{frame}

%%%%%%%%%%%%%%%%%%%%%%%%%%%%%%%%%%%%%%%%%%%%%%%%%%%%%%%%%%%%%%%%%%%%%%%%%%%%%%%%

\begin{frame}

\implibtitle

\bookpath{kestrel/apt}:
Automated Program Transformations (APT).
\begin{itemize}
\item Improved the \code{simplify} transformation.
\item Improved the robustness of the proofs
      generated by the \code{restrict} transformation.
\end{itemize}

\end{frame}

%%%%%%%%%%%%%%%%%%%%%%%%%%%%%%%%%%%%%%%%%%%%%%%%%%%%%%%%%%%%%%%%%%%%%%%%%%%%%%%%

\begin{frame}

\implibtitle

\bookpath{kestrel/c}:
Models, proofs, and tools for C.
\begin{itemize}
\item Refactored to organize constituents
      (deep embedding, shallow embedding, code generator)
      more clearly.
\item Extensions and improvements to the model of C.
\item Significant extensions of the C code generation capabilities.
\item Significant performance improvements of the generated correctness proofs,
      via a new modular proof approach.
\end{itemize}

\end{frame}

%%%%%%%%%%%%%%%%%%%%%%%%%%%%%%%%%%%%%%%%%%%%%%%%%%%%%%%%%%%%%%%%%%%%%%%%%%%%%%%%

\begin{frame}

\implibtitle

\bookpath{kestrel/crypto/pfcs}:
Prime Field Constraint Systems (PFCS).
(Discussed in a paper at this Workshop.)
\begin{itemize}
\item Added a concrete syntax.
\item Improved the abstract syntax.
\item Added many theorems,
      including proof support rules for compositional reasoning.
\item Added a translator from R1CS to PFCS,
      along with a checker for the R1CS subset of PFCS.
\item Added a proof-generating lifter from deeply to shallowly embedded PFCS.
\end{itemize}

\end{frame}

%%%%%%%%%%%%%%%%%%%%%%%%%%%%%%%%%%%%%%%%%%%%%%%%%%%%%%%%%%%%%%%%%%%%%%%%%%%%%%%%

\begin{frame}

\implibtitle

\bookpath{kestrel/std/system}:
Standard system library (extension of Std).
\begin{itemize}
\item Added utilities
      \code{untranslate\$},
      \code{genvar\$},
      \code{one-way-unify\$}, and
      \code{termination-theorem\$},
      which are logic-mode variants of built-in utilities,
      via \code{magic-ev-fncall}.
\item Added utilities
      \code{guard-theorem-no-simplify} (program-mode) and
      \code{guard-theorem-no-simplify\$} (logic-mode),
      which are variants of \code{guard-theorem} that does no simplification.
\end{itemize}

\end{frame}

%%%%%%%%%%%%%%%%%%%%%%%%%%%%%%%%%%%%%%%%%%%%%%%%%%%%%%%%%%%%%%%%%%%%%%%%%%%%%%%%

\begin{frame}

\implibtitle

\bookpath{kestrel/std/util}:
Standard system library (extension of Std).
\begin{itemize}
\item New utilities for error-value tuples have been added,
      to facilitate the generation, propagation, and catching of errors
      in statically strongly typed code.
\end{itemize}

\end{frame}

%%%%%%%%%%%%%%%%%%%%%%%%%%%%%%%%%%%%%%%%%%%%%%%%%%%%%%%%%%%%%%%%%%%%%%%%%%%%%%%%

\begin{frame}

\implibtitle

\bookpath{kestrel/utilities/digits-any-base}:
Number representations in arbitrary bases.
\begin{itemize}
\item Added several theorems.
\item Extended the \code{defdigits} macro to generate more theorems.
\end{itemize}

\end{frame}

%%%%%%%%%%%%%%%%%%%%%%%%%%%%%%%%%%%%%%%%%%%%%%%%%%%%%%%%%%%%%%%%%%%%%%%%%%%%%%%%

\begin{frame}

\implibtitle

\bookpath{kestrel/utilities/omaps}:
Ordered maps (omaps).
\begin{itemize}
\item Added several theorems.
\item Improved documentation and organization.
\end{itemize}

\end{frame}

%%%%%%%%%%%%%%%%%%%%%%%%%%%%%%%%%%%%%%%%%%%%%%%%%%%%%%%%%%%%%%%%%%%%%%%%%%%%%%%%

\begin{frame}

\implibtitle

\bookpath{std/util}:
Standard utilities library.
\begin{itemize}
\item The \code{er} binder of \code{b*} has been extended with
      an option \code{:iferr} to return an alternative value in case of error.
\end{itemize}

\end{frame}

%%%%%%%%%%%%%%%%%%%%%%%%%%%%%%%%%%%%%%%%%%%%%%%%%%%%%%%%%%%%%%%%%%%%%%%%%%%%%%%%

\end{document}
