% GENERAL FORMAT OF EACH SLIDE
% ----------------------------

%\begin{frame}

%\newlibtitle or \implibtitle % for new vs. improved libraries

%\bookpath{path/to/the/library}: % directory or file, use macro for font
% Short description of what this library is, 1-2 lines.
%\begin{itemize}
%\item
% A highlight about this new or improved library.
%\item
% Another highlight.
%\item
% Say if it's described in a paper at this workshop,
% in which case, there may be little to say here.
%\end{itemize}

%\separation % to separate the next entry, if any

%\bookpath{another/path}: % as above

%\end{frame}

% ORDER OF THE SLIDES AND ENTRIES
% -------------------------------

% First all the new libraries, in alphabetical order.

% Then all the improved libraries, in alphabetical order.

% Then one or more slides with additional contributions.

% Don't worry about adding your contributions in order:
% Alessandro will quickly put them in order if needed,
% and also fix any formatting issues.

% TITLE SLIDE
% -----------

% Add your name as co-author, in alphabetical order of last name.

% Add your organization, in alphabetical order.
% Renumber the \inst{n} instances as needed
% (both inside \author{...} and inside \institute{...}).

% Or ask Alessandro to do this.

% OTHER REMARKS
% -------------

% Use the \code{...} macro for code (i.e.\ fixed-width) font within the text.

% If you want to show code block snippets,
% use the verbatim environment,
% and add [fragile] just after \begin{frame}.

% For guidelines on content and level of detail,
% please see existing entries.
% The goal is to provide enough information to get possible users interested.

%%%%%%%%%%%%%%%%%%%%%%%%%%%%%%%%%%%%%%%%%%%%%%%%%%%%%%%%%%%%%%%%%%%%%%%%%%%%%%%%

\documentclass{beamer}
\usetheme{Singapore}
\usecolortheme{orchid}
\usepackage{inconsolata}

\beamertemplatenavigationsymbolsempty

\setbeamertemplate{footline}[frame number]

%%%%%%%%%%%%%%%%%%%%%%%%%%%%%%%%%%%%%%%%%%%%%%%%%%%%%%%%%%%%%%%%%%%%%%%%%%%%%%%%

\newcommand{\code}[1]{\texttt{#1}}
\newenvironment{codeblock}{\begin{alltt}}{\end{alltt}}

\newcommand{\bookpath}[1]{\textbf{\code{#1}}}

\newcommand{\newlibtitle}{\frametitle{New Libraries}}
\newcommand{\implibtitle}{\frametitle{Improved Libraries}}

\newcommand{\separation}{\vspace*{1.5ex}}

%%%%%%%%%%%%%%%%%%%%%%%%%%%%%%%%%%%%%%%%%%%%%%%%%%%%%%%%%%%%%%%%%%%%%%%%%%%%%%%%

\title{What's New in the Community Books}

\subtitle{Since the ACL2-2020 Workshop}

\author{Jagadish Bapanapally\inst{9}\inst{5},
        Cuong Chau\inst{2},\\
        Alessandro Coglio\inst{1}\inst{5}\inst{6} (presenter),
        Shilpi Goel\inst{3},\\
        Matt Kaufmann\inst{8},
        Panagiotis Manolios\inst{7},\\
        Eric McCarthy\inst{1}\inst{5},
        J Strother Moore\inst{8},\\
        David Russinoff\inst{2},
        Eric W. Smith\inst{1}\inst{5}\inst{6},\\
        Sol Swords\inst{4},
        Stephen Westfold\inst{5}}

\institute{\inst{1}Aleo,
           \inst{2}ARM,
           \inst{3}Centaur,
           \inst{4}Intel,\\
           \inst{5}Kestrel Institute,
           \inst{6}Kestrel Technology,\\
           \inst{7}Northeastern University,\\
           \inst{8}University of Texas at Austin (retired),\\
           \inst{9}University of Wyoming}

\date{ACL2-2022 Workshop}

%%%%%%%%%%%%%%%%%%%%%%%%%%%%%%%%%%%%%%%%%%%%%%%%%%%%%%%%%%%%%%%%%%%%%%%%%%%%%%%%

\begin{document}

%%%%%%%%%%%%%%%%%%%%%%%%%%%%%%%%%%%%%%%%%%%%%%%%%%%%%%%%%%%%%%%%%%%%%%%%%%%%%%%%

\frame[noframenumbering,plain]{\titlepage}

%%%%%%%%%%%%%%%%%%%%%%%%%%%%%%%%%%%%%%%%%%%%%%%%%%%%%%%%%%%%%%%%%%%%%%%%%%%%%%%%

\begin{frame}

\frametitle{Overview}

\begin{itemize}
\item Over 8,000 non-merge commits since the last Workshop.
\item From several contributors from several organizations.
\item Spanning hardware, mathematics, cryptography, blockchain,
      programming languages, virtual machines, machine code,
      standards, analysis, synthesis, and more.
\item These slides provides
      a more succinct overview than the book release notes,
      ordered by book path within each of the new and improved library parts.
\end{itemize}

\end{frame}

%%%%%%%%%%%%%%%%%%%%%%%%%%%%%%%%%%%%%%%%%%%%%%%%%%%%%%%%%%%%%%%%%%%%%%%%%%%%%%%%

\begin{frame}[fragile]

\newlibtitle

\bookpath{centaur/bigmem}:
A library that defines a big array (up to 2$^{64}$ bytes).
\begin{itemize}
\item Offers the reasoning efficiency of records.
\item Offers efficient execution via nested stobjs containing resizable arrays.
\end{itemize}

\end{frame}

%%%%%%%%%%%%%%%%%%%%%%%%%%%%%%%%%%%%%%%%%%%%%%%%%%%%%%%%%%%%%%%%%%%%%%%%%%%%%%%%

\begin{frame}

\newlibtitle

\bookpath{kestrel/acl2-arrays}:
Support for reasoning about programs that use ACL2 arrays (e.g.,
values satisfying \code{array1p}).

\begin{itemize}
\item Many rules about existing functions.
\item New operations that make arrays exapandable.
\item Tool for defining typed ACL2 arrays.
\end{itemize}

\end{frame}

%%%%%%%%%%%%%%%%%%%%%%%%%%%%%%%%%%%%%%%%%%%%%%%%%%%%%%%%%%%%%%%%%%%%%%%%%%%%%%%%

\begin{frame}[fragile]

\newlibtitle

\bookpath{kestrel/acl2pl}:
Preliminary model of the ACL2 programming (not logical) language.
\begin{itemize}
\item Abstract syntax consisting of translated terms, functions, and packages.
\item Small-step evaluation semantics.
\item Program-mode interpreter.
\item Lifter from packages and functions in the world to the formal model.
\end{itemize}

\end{frame}

%%%%%%%%%%%%%%%%%%%%%%%%%%%%%%%%%%%%%%%%%%%%%%%%%%%%%%%%%%%%%%%%%%%%%%%%%%%%%%%%
\begin{frame}

\newlibtitle

\bookpath{kestrel/algeba}:
A library for abstarct algebra
\begin{itemize}
\item Add a formalization of groups.
\item Prove some simple properties, in a calculational style.
\end{itemize}

\end{frame}

%%%%%%%%%%%%%%%%%%%%%%%%%%%%%%%%%%%%%%%%%%%%%%%%%%%%%%%%%%%%%%%%%%%%%%%%%%%%%%%%

\begin{frame}

\newlibtitle

\bookpath{kestrel/algorithm-theories}:
A library to collect algorithm schemes.
\begin{itemize}
\item Scheme for a generic tail-recursive function.
\end{itemize}

\end{frame}

%%%%%%%%%%%%%%%%%%%%%%%%%%%%%%%%%%%%%%%%%%%%%%%%%%%%%%%%%%%%%%%%%%%%%%%%%%%%%%%%

\begin{frame}

\newlibtitle

\bookpath{kestrel/arrays-2d}:
A formalization of two-dimensional arrays as lists of lists.
\begin{itemize}
\item Supports arrays with arbitrary elements, and with elements that
  are known to be bit-vectors.
\end{itemize}

\end{frame}

%%%%%%%%%%%%%%%%%%%%%%%%%%%%%%%%%%%%%%%%%%%%%%%%%%%%%%%%%%%%%%%%%%%%%%%%%%%%%%%%

\begin{frame}

\newlibtitle

\bookpath{kestrel/c}:
Models, proofs, and tools for C (described in a paper at this workshop).
\begin{itemize}
\item ATC, the C code generator for ACL2.
\item Deep embedding of C in ACL2.
\item Shallow embedding of C in ACL2.
\end{itemize}

\end{frame}

%%%%%%%%%%%%%%%%%%%%%%%%%%%%%%%%%%%%%%%%%%%%%%%%%%%%%%%%%%%%%%%%%%%%%%%%%%%%%%%%

\begin{frame}

\newlibtitle

\bookpath{kestrel/clause-processors}:
Modular collection of clause-processors.
\begin{itemize}
\item Collect several clause-processors (subst, flatten literals, simple subsumption), many of which are verified.
\end{itemize}

\end{frame}

%%%%%%%%%%%%%%%%%%%%%%%%%%%%%%%%%%%%%%%%%%%%%%%%%%%%%%%%%%%%%%%%%%%%%%%%%%%%%%%%

\begin{frame}

\newlibtitle

\bookpath{kestrel/crypto/blake}: BLAKE Library.
\begin{itemize}
\item Formal specifications of the BLAKE2s, BLAKE2s-extended, and
  BLAKE-256 hash functions.
\end{itemize}

\end{frame}

%%%%%%%%%%%%%%%%%%%%%%%%%%%%%%%%%%%%%%%%%%%%%%%%%%%%%%%%%%%%%%%%%%%%%%%%%%%%%%%%

\begin{frame}

\newlibtitle

\bookpath{kestrel/crypto/salsa}:
Formal spec of the Salsa20 hash function.

\end{frame}

%%%%%%%%%%%%%%%%%%%%%%%%%%%%%%%%%%%%%%%%%%%%%%%%%%%%%%%%%%%%%%%%%%%%%%%%%%%%%%%%

\begin{frame}

\newlibtitle

\bookpath{kestrel/crypto/mimc}:
Minimal Multiplicative Complexity (MiMC) hash function.
\begin{itemize}
\item Formalization of the MiMC hash function.
\item Uses a sponge construction.
\item Used in Ethereum's Semaphore zero-knowledge gadget.
\end{itemize}

\end{frame}

%%%%%%%%%%%%%%%%%%%%%%%%%%%%%%%%%%%%%%%%%%%%%%%%%%%%%%%%%%%%%%%%%%%%%%%%%%%%%%%%

\begin{frame}

\newlibtitle

\bookpath{kestrel/crypto/pfcs}:
Prime Field Constraint Systems (PFCS).
\begin{itemize}
\item Generalization of Rank-1 Constraint Systems (R1CS).
\item Formal syntax and semantics.
\item Preliminary reasoning support.
\item Useful in zero-knowledge cryptographic proofs.
\end{itemize}

\end{frame}

%%%%%%%%%%%%%%%%%%%%%%%%%%%%%%%%%%%%%%%%%%%%%%%%%%%%%%%%%%%%%%%%%%%%%%%%%%%%%%%%

\begin{frame}

\newlibtitle

\bookpath{kestrel/crypto/r1cs}: R1CS Library.
\begin{itemize}
\item A formal semantics for rank-1 constraint systems (R1CSes). These
  are often used in zero-knowledge proofs.
\item Extensive support for reasoning about R1CSes (using ACL2 and Axe).
\item Verified R1CS gadgets / gadget generators.
\end{itemize}

\end{frame}

%%%%%%%%%%%%%%%%%%%%%%%%%%%%%%%%%%%%%%%%%%%%%%%%%%%%%%%%%%%%%%%%%%%%%%%%%%%%%%%%

\begin{frame}

\newlibtitle

\bookpath{kestrel/evaluators}:
Simple evaluators for verifying clause-processors and metafunctions.
\begin{itemize}
\item Several evaluations for common sets of functions.
\item New defevaluator+ tool (better defaults, extra/better theorems,
      support for lifting results to richer evaluators).
\end{itemize}

\end{frame}

%%%%%%%%%%%%%%%%%%%%%%%%%%%%%%%%%%%%%%%%%%%%%%%%%%%%%%%%%%%%%%%%%%%%%%%%%%%%%%%%

\begin{frame}

\newlibtitle

\bookpath{kestrel/isar}:
Tools for Isar-style proofs in ACL2.
\begin{itemize}
\item Isar = Intelligible semi-automated reasoning
      (proof language of the Isabelle theorem prover).
\item Human-oriented (vs.\ machine-oriented) readable proofs.
\item \code{:assume} ... \code{:let} ... \code{:derive} ... \code{:qed}.
\item Useful, for instance, in proofs involving algebraic manipulations
      that do not follow simple rewriting directions.
\item This library is just a small start.
\end{itemize}

\end{frame}

%%%%%%%%%%%%%%%%%%%%%%%%%%%%%%%%%%%%%%%%%%%%%%%%%%%%%%%%%%%%%%%%%%%%%%%%%%%%%%%%

\begin{frame}

\newlibtitle

\bookpath{kestrel/json}:
Models and tools for JSON.
\begin{itemize}
\item Abstract syntax of JSON.
\item Converter from parser's abstract syntax to this abstract syntax.
\item Includes b* binder for pattern matching on JSON structures.
\end{itemize}

\end{frame}

%%%%%%%%%%%%%%%%%%%%%%%%%%%%%%%%%%%%%%%%%%%%%%%%%%%%%%%%%%%%%%%%%%%%%%%%%%%%%%%%

\begin{frame}

\newlibtitle

\bookpath{kestrel/json-parser}:
A JSON parser implemented in ACL2.
\begin{itemize}
\item Unicode support.
\end{itemize}

\end{frame}

%%%%%%%%%%%%%%%%%%%%%%%%%%%%%%%%%%%%%%%%%%%%%%%%%%%%%%%%%%%%%%%%%%%%%%%%%%%%%%%%

\begin{frame}

\newlibtitle

\bookpath{kestrel/jvm}:
Kestrel's model of the Java Virtual Machine.
\begin{itemize}
\item Support for code proofs and lifting.
\item Works with Kestrel's Axe toolkit.
\item Includes a class file parser.
\end{itemize}

\end{frame}

%%%%%%%%%%%%%%%%%%%%%%%%%%%%%%%%%%%%%%%%%%%%%%%%%%%%%%%%%%%%%%%%%%%%%%%%%%%%%%%%

\begin{frame}

\newlibtitle

\bookpath{kestrel/number-theory}:
A library about number theory.
\begin{itemize}
\item Primality, divisibility, quadratic residues, etc.
\item Includes the \code{defprime} and \code{defprime-alias} tools to
  introduce standard reasoning machinery for primes.
\item Includes verified Tonelli-Shanks algorithm for modular square roots.
\end{itemize}

\end{frame}

%%%%%%%%%%%%%%%%%%%%%%%%%%%%%%%%%%%%%%%%%%%%%%%%%%%%%%%%%%%%%%%%%%%%%%%%%%%%%%%%

\begin{frame}

\newlibtitle

\bookpath{kestrel/random}:
A lightweight library containing some simple random number generators.

\end{frame}

%%%%%%%%%%%%%%%%%%%%%%%%%%%%%%%%%%%%%%%%%%%%%%%%%%%%%%%%%%%%%%%%%%%%%%%%%%%%%%%%

\begin{frame}

\newlibtitle

\bookpath{kestrel/sequences}:
A library for defining higher-order operations over lists.
\begin{itemize}
\item \code{defforall}
\item \code{defexists}
\item \code{defmap}
\item \code{deffilter}
\end{itemize}
\end{frame}

%%%%%%%%%%%%%%%%%%%%%%%%%%%%%%%%%%%%%%%%%%%%%%%%%%%%%%%%%%%%%%%%%%%%%%%%%%%%%%%%

\begin{frame}

\newlibtitle

\bookpath{kestrel/simpl-imp}:
A simple didactic programming language, Imp.
\begin{itemize}
\item Formal syntax and semantics.
\item Useful for didactic purposes.
\item Used in the literature.
\end{itemize}

\end{frame}

%%%%%%%%%%%%%%%%%%%%%%%%%%%%%%%%%%%%%%%%%%%%%%%%%%%%%%%%%%%%%%%%%%%%%%%%%%%%%%%%

\begin{frame}

\newlibtitle

\bookpath{kestrel/solidity}:
Small start towards a model of Solidity,
the main smart contract language for Ethereum.
\begin{itemize}
\item A model of (some) Solidity values.
\end{itemize}

\end{frame}

%%%%%%%%%%%%%%%%%%%%%%%%%%%%%%%%%%%%%%%%%%%%%%%%%%%%%%%%%%%%%%%%%%%%%%%%%%%%%%%%

\begin{frame}

\newlibtitle

\bookpath{kestrel/strings-light}:
A lightweight library about strings
\begin{itemize}
\item Changing case.
\item Splitting strings and character lists.
\item Checking suffixes.
\item Books about reverse and length.
\item Parsing chars as digits (lightweight)
\end{itemize}


\end{frame}

%%%%%%%%%%%%%%%%%%%%%%%%%%%%%%%%%%%%%%%%%%%%%%%%%%%%%%%%%%%%%%%%%%%%%%%%%%%%%%%%

\begin{frame}

\newlibtitle

\bookpath{kestrel/syntheto}:
Models, proofs, and tools for Syntheto (described in a paper at this workshop).
\begin{itemize}
\item Surface language for APT and ACL2.
\item Formalization of abstract syntax.
\item Formalization of static semantics.
\item Translation to ACL2 (Syntheto back end).
\end{itemize}

\end{frame}

%%%%%%%%%%%%%%%%%%%%%%%%%%%%%%%%%%%%%%%%%%%%%%%%%%%%%%%%%%%%%%%%%%%%%%%%%%%%%%%%

\begin{frame}

\newlibtitle

\bookpath{kestrel/terms-light}:
A lightweight library of operations on terms.
\begin{itemize}
\item Find free and bound vars.  Rename vars.
\item Substitution and evaluation.
\item Add/remove/serialize \code{let}s / \code{lambda}s.
\item Reconstruct \code{mv-let}s.
\item Check terms (closed lambdas, no duplicate lambda vars, \code{nil} not used
      as a var).
\item Search, count, create, and transform terms.
\end{itemize}

\end{frame}

%%%%%%%%%%%%%%%%%%%%%%%%%%%%%%%%%%%%%%%%%%%%%%%%%%%%%%%%%%%%%%%%%%%%%%%%%%%%%%%%

\begin{frame}

\newlibtitle

\bookpath{kestrel/typed-lists-light}:
A lightweight library dealing with lists of objects of particular
types (rather than lists in general).
\begin{itemize}
\item lists of integers, lists of symbols, lists of pseudo-terms, etc.
\end{itemize}

\end{frame}

%%%%%%%%%%%%%%%%%%%%%%%%%%%%%%%%%%%%%%%%%%%%%%%%%%%%%%%%%%%%%%%%%%%%%%%%%%%%%%%%

\begin{frame}

\newlibtitle

\bookpath{kestrel/unicode-light}:
A lightweight library about Unicode.
\begin{itemize}
\item UTF-8 encoding.
\item UTF-16 surrogate code points.
\end{itemize}

\end{frame}

%%%%%%%%%%%%%%%%%%%%%%%%%%%%%%%%%%%%%%%%%%%%%%%%%%%%%%%%%%%%%%%%%%%%%%%%%%%%%%%%

\begin{frame}

\newlibtitle

\bookpath{kestrel/untranslated-terms}:
A new library for manipulating untranslated terms.
\begin{itemize}
\item Allows structure to be maintained that would be lost by translation.
\end{itemize}

\end{frame}

%%%%%%%%%%%%%%%%%%%%%%%%%%%%%%%%%%%%%%%%%%%%%%%%%%%%%%%%%%%%%%%%%%%%%%%%%%%%%%%%

\begin{frame}

\newlibtitle

\bookpath{kestrel/x86}:
Kestrel's x86 proof machinery, which complements the X86ISA model.
\begin{itemize}
\item Focus on readability of proof terms.
\item Supports the lifting of x86 code into logic with the Axe toolkit.
\item Includes parsers for PE and Mach-O executables.
\end{itemize}

\end{frame}

%%%%%%%%%%%%%%%%%%%%%%%%%%%%%%%%%%%%%%%%%%%%%%%%%%%%%%%%%%%%%%%%%%%%%%%%%%%%%%%%

\begin{frame}

\newlibtitle

\bookpath{kestrel/yul}:
Models, proofs, and tools for Yul,
an intermediate language used in the Solidity compiler.
\begin{itemize}
\item Concrete syntax and parser.
\item Abstract syntax.
\item Static semantics.
\item Dynamic semantics.
\item Static soundness proof.
\item Covers all of `generic Yul'.
\item Some verified Yul-to-Yul transformations (used in the Solidity compiler).
\end{itemize}

\end{frame}

%%%%%%%%%%%%%%%%%%%%%%%%%%%%%%%%%%%%%%%%%%%%%%%%%%%%%%%%%%%%%%%%%%%%%%%%%%%%%%%%

\begin{frame}

\newlibtitle

\bookpath{kestrel/zcash}:
Models and proofs for the Zcash blockchain.
\begin{itemize}
\item Formalization of some zero-knowledge-related operations.
\item Formalization and verification of some R1CS gadgets /
      gadget generators.
\item New \code{verify-zcash-r1cs} tool based on Axe.
\end{itemize}
\end{frame}

%%%%%%%%%%%%%%%%%%%%%%%%%%%%%%%%%%%%%%%%%%%%%%%%%%%%%%%%%%%%%%%%%%%%%%%%%%%%%%%%

\begin{frame}

\newlibtitle

\bookpath{projects/execloader}:
Binary loaders.
\begin{itemize}
\item Read in sections of ELF/Mach-O files into ACL2. An older version
      of these books used to live in the \code{x86isa} library.
\item Simplified \code{elf-reader}; ELF binary header, all section
      headers, and all section contents are now stored in the \code{elf}
      stobj. Previously, only a handful of commonly-used sections (e.g.,
      \code{.text}, \code{.data}, \code{.rodata}, etc.) were parsed.
\item Added support for getting information from ELF symbol table
      using functions \code{get-symtab-entries} and
      \code{get-label-address}.
\end{itemize}
\end{frame}

%%%%%%%%%%%%%%%%%%%%%%%%%%%%%%%%%%%%%%%%%%%%%%%%%%%%%%%%%%%%%%%%%%%%%%%%%%%%%%%%

\begin{frame}

\newlibtitle

\bookpath{std/obags}:
Ordered bags (obags).
\begin{itemize}
\item Similar to osets and omaps, for bags (i.e.\ multisets).
\item Modeled as totally (non-strictly) ordered lists.
\item Include operations and theorems.
\end{itemize}

\end{frame}

%%%%%%%%%%%%%%%%%%%%%%%%%%%%%%%%%%%%%%%%%%%%%%%%%%%%%%%%%%%%%%%%%%%%%%%%%%%%%%%%

\begin{frame}

\implibtitle

\bookpath{acl2s}: ACL2s Sedan.
\begin{itemize}
\item Added \code{properties} book with \code{definec}-like syntax
  supporting property-based design, testing and verification.
\item Various improvements to \code{definec} and \code{defunc}
  regarding performance, debugging support, extensions.
\item Updated utilities, such as \code{acl2s::match} macro, for pattern matching.
\item More polymorphic support, built-in types, alias types, etc.
  in \code{defdata}.
\item Improvements for counterexample generations with \code{cgen},
  including using fixers in ACL2s.
\end{itemize}

\end{frame}

%%%%%%%%%%%%%%%%%%%%%%%%%%%%%%%%%%%%%%%%%%%%%%%%%%%%%%%%%%%%%%%%%%%%%%%%%%%%%%%%

\begin{frame}

\implibtitle

\bookpath{arithmetic}:
Arithmetic library.
\begin{itemize}
\item Reduced what is exported.
\end{itemize}

\end{frame}

%%%%%%%%%%%%%%%%%%%%%%%%%%%%%%%%%%%%%%%%%%%%%%%%%%%%%%%%%%%%%%%%%%%%%%%%%%%%%%%%

\begin{frame}

\implibtitle

\bookpath{build}:
Build system.
\begin{itemize}
\item Now \code{cert.pl} makes use of useless runes just as \code{make} does.
\item Now \code{make} of the community books also writes book dependency
  information in S-expression form.
\item Swapped the roles of green and bold green in build output (bold for
      slower books).
\end{itemize}

\end{frame}

%%%%%%%%%%%%%%%%%%%%%%%%%%%%%%%%%%%%%%%%%%%%%%%%%%%%%%%%%%%%%%%%%%%%%%%%%%%%%%%%

\begin{frame}

\implibtitle

\bookpath{centaur/defrstobj2}:
Record-like stobjs.
\begin{itemize}
\item \code{defrstobj2} can now be used to define stobjs with child
      stobj fields, i.e., fields based on another stobj.
\end{itemize}

\end{frame}

%%%%%%%%%%%%%%%%%%%%%%%%%%%%%%%%%%%%%%%%%%%%%%%%%%%%%%%%%%%%%%%%%%%%%%%%%%%%%%%%

\begin{frame}

\implibtitle

\bookpath{centaur/fgl}:
FGL symbolic execution engine.
\begin{itemize}
\item Added incremental-minimize/maximize and minimize/maximize-ratio tools.
\end{itemize}
\end{frame}

%%%%%%%%%%%%%%%%%%%%%%%%%%%%%%%%%%%%%%%%%%%%%%%%%%%%%%%%%%%%%%%%%%%%%%%%%%%%%%%%

\begin{frame}

\implibtitle

\bookpath{centaur/sv}:
Hardware verification backend.
\begin{itemize}
\item New flow for producing a symbolic unrolling (SVTV).
\item More complete logical story for process
      from hierarchical design to unrolling.
\item New utilities for proof (de)composition (\code{def-svtv-override-fact}).
\item Improvements to \code{SVTV-CHASE} utility.
\end{itemize}

\end{frame}

%%%%%%%%%%%%%%%%%%%%%%%%%%%%%%%%%%%%%%%%%%%%%%%%%%%%%%%%%%%%%%%%%%%%%%%%%%%%%%%%

\begin{frame}

\implibtitle

\bookpath{centaur/vl}:
SystemVerilog frontend.
\begin{itemize}
\item New support for SystemVerilog calls of
      static methods of parametrized classes.
\item Improved preprocessor performance when there are lots of \code{define}s.
\item Reduce deps of \code{vl/util/namedb}, used by \code{defrstobj} and the x86
      model.
\end{itemize}

\end{frame}

%%%%%%%%%%%%%%%%%%%%%%%%%%%%%%%%%%%%%%%%%%%%%%%%%%%%%%%%%%%%%%%%%%%%%%%%%%%%%%%%

\begin{frame}

\implibtitle

\bookpath{doc}:
Documentation.
\begin{itemize}
\item Web-based manual now includes clickable links to GitHub for doc topics
      from the Community Books.
\item The file \code{[books]/doc/top-slow.lisp} is now \code{[books]/top.lisp}
      and no longer builds a manual.  Instead, it detects name conflicts between
      community books.
\item Manual building is now taken care of by [books]/doc/top.lisp.
\end{itemize}

\end{frame}

%%%%%%%%%%%%%%%%%%%%%%%%%%%%%%%%%%%%%%%%%%%%%%%%%%%%%%%%%%%%%%%%%%%%%%%%%%%%%%%%

\begin{frame}

\implibtitle

\bookpath{kestrel/abnf}:
Augmented Backus-Naur Form (ABNF).
\begin{itemize}
\item Refactored to move parser verification proof to separate file from parser.
\item Refactored to collect parsing primitives usable in other parsers.
\item Added preliminary parsing generation tools.
\end{itemize}

\end{frame}

%%%%%%%%%%%%%%%%%%%%%%%%%%%%%%%%%%%%%%%%%%%%%%%%%%%%%%%%%%%%%%%%%%%%%%%%%%%%%%%%

\begin{frame}

\implibtitle

\bookpath{kestrel/alists-light}:
Lightweight alists library.
\begin{itemize}
\item New rules and books, including books on \code{alistp},
  \code{clear-key}, \code{rassoc-equal}, and \code{keep-pairs}.
\item Improve modularity.
\end{itemize}

\end{frame}

%%%%%%%%%%%%%%%%%%%%%%%%%%%%%%%%%%%%%%%%%%%%%%%%%%%%%%%%%%%%%%%%%%%%%%%%%%%%%%%%

\begin{frame}

\implibtitle

\bookpath{kestrel/apt}:
Automated Program Transformations (APT).
\begin{itemize}
\item Added new \code{schemalg} transformation to apply algorithm schemas,
      currently supporting divide-and-conquer schemas.
\item Added new \code{solve} transformation to attempt to solve a specification,
      currently supporting the ACL2 and Axe rewriters as solvers.
\item Added new \code{expdata} transformation to refine data types
      where each instance of the old data may be represented byte
      multiple instances of the new data (i.e\ not isomorphic).
\item Added new \code{drop-irrelevant-params} and \code{rename-params} transformations.
\item Added new \code{wrap-output} transformation to change a function's return type.
\item Added new \code{finite-difference} transformation for incrementalization.
\item Added new \code{copy-function} example transformation.
\item Improved and extended existing
      \code{isodata}, \code{restrict}, \code{simplify}, and \code{tailrec}
      transformations.
\item New tools, \code{deftransformation} and \code{def-equality-transformation}, to generate transformations, handling
      the boilerplate.
\end{itemize}

\end{frame}

%%%%%%%%%%%%%%%%%%%%%%%%%%%%%%%%%%%%%%%%%%%%%%%%%%%%%%%%%%%%%%%%%%%%%%%%%%%%%%%%

\begin{frame}

\implibtitle

\bookpath{kestrel/arithmetic-light}:
Lightweight arithmetic library.
\begin{itemize}
\item Many new rules and books have been added, including books on
  \code{integer-length} \code{ceiling-of-lg}, evenness and oddness,
  \code{truncate}, \code{rem}, \code{ash}, \code{min}, \code{max},
  \code{<=}, \code{abs}, and \code{natp}.
\end{itemize}

\end{frame}

%%%%%%%%%%%%%%%%%%%%%%%%%%%%%%%%%%%%%%%%%%%%%%%%%%%%%%%%%%%%%%%%%%%%%%%%%%%%%%%%

\begin{frame}

\implibtitle

\bookpath{kestrel/axe}:
Axe Toolkit.
\begin{itemize}
\item Major additions; most of Axe is now open-source.
\item Machinery for making customized Axe rewriters and provers.
\item A new general-purpose prover and rewriter
      (guard-verified, \code{:logic}-mode code).
\item Legacy prover and rewriter.
\item Axe Tactic Prover.
\item Axe Equivalence Checker.
\item Connection to the STP SMT solver.
\item Tools to expand/unroll/simplify specifications.
\item Many rules useful in Axe proofs.
\item Utilities to unroll specifications through rewriting.
\end{itemize}

\end{frame}

%%%%%%%%%%%%%%%%%%%%%%%%%%%%%%%%%%%%%%%%%%%%%%%%%%%%%%%%%%%%%%%%%%%%%%%%%%%%%%%%

\begin{frame}

\implibtitle

\bookpath{kestrel/axe/jvm}:
Axe toolkit for JVM.
\begin{itemize}
\item Tools to lift JVM code into logic.
\item Formal Unit Tester tool, for small solver-backed proofs about
bounded executions of programs.
\end{itemize}

\end{frame}

%%%%%%%%%%%%%%%%%%%%%%%%%%%%%%%%%%%%%%%%%%%%%%%%%%%%%%%%%%%%%%%%%%%%%%%%%%%%%%%%

\begin{frame}

\implibtitle

\bookpath{kestrel/axe/r1cs}:
Axe toolkit for R1CS (rank-1 constraint systems).
\begin{itemize}
\item Tools to lift R1CSes into logic and verify them.
\end{itemize}

\end{frame}

%%%%%%%%%%%%%%%%%%%%%%%%%%%%%%%%%%%%%%%%%%%%%%%%%%%%%%%%%%%%%%%%%%%%%%%%%%%%%%%%

\begin{frame}

\implibtitle

\bookpath{kestrel/axe/x86}:
Axe toolkit for x86.
\begin{itemize}
\item Tools to lift x86 code into logic.
\end{itemize}

\end{frame}

%%%%%%%%%%%%%%%%%%%%%%%%%%%%%%%%%%%%%%%%%%%%%%%%%%%%%%%%%%%%%%%%%%%%%%%%%%%%%%%%

\begin{frame}

\implibtitle

\bookpath{kestrel/bitcoin}:
Bitcoin library.
\begin{itemize}
\item Added formalization of the Bech32 and Bech32m checksummed base32 formats
      used to encode addresses in Segwit.
\end{itemize}

\end{frame}

%%%%%%%%%%%%%%%%%%%%%%%%%%%%%%%%%%%%%%%%%%%%%%%%%%%%%%%%%%%%%%%%%%%%%%%%%%%%%%%%

\begin{frame}

\implibtitle

\bookpath{kestrel/booleans}:
Booleans library.
\begin{itemize}
\item New rules and \code{defcong}s.
\end{itemize}

\end{frame}

%%%%%%%%%%%%%%%%%%%%%%%%%%%%%%%%%%%%%%%%%%%%%%%%%%%%%%%%%%%%%%%%%%%%%%%%%%%%%%%%

\begin{frame}

\implibtitle

\bookpath{kestrel/bv}:
BV (bit-vector) library.
\begin{itemize}
\item Over 1000 new rules.
\item New books have been added covering many more BV operations,
including subtraction, arithmetic negation, multiplicaton, shifts,
bitwise \code{OR} and \code{AND}, logical negation, signed and
unsigned comparisons, signed and unsigned division and remainder,
trimming, sign extension, various single-bit operations,
bit-vector-valued conditionals, converting between bits and booleans,
recognizing bits and (signed and unsigned) bytes, repeating a bit, and
counting the number of 1 bits.
\item Rules to characterize signed addition overflow and underflow.
\item Rules to turn BV ops into more common or more idiomatic operations.
\item A formalization of one's complement numbers and addition.
\item Various syntactic functions over BV-valued terms.
\end{itemize}

\end{frame}

%%%%%%%%%%%%%%%%%%%%%%%%%%%%%%%%%%%%%%%%%%%%%%%%%%%%%%%%%%%%%%%%%%%%%%%%%%%%%%%%

\begin{frame}

\implibtitle

\bookpath{kestrel/bv-lists}:
BV-Lists library.

\begin{itemize}
\item Many new rules.
\item New books about \code{bv-arrayp}, \code{bv-array-read},
\code{bv-array-write}, \code{all-all-unsigned-byte-p},
\code{width-of-widest-int}, \code{bvnot-list}, \code{getbit-list},
\code{map-slice}, \code{bvplus-list}, \code{logext-list},
\code{bv-nth}, \code{map-packbv}, \code{all-signed-byte-p},
conversions between lists and bv-arrays, \code{packbv-little}, and
\code{byte-listp}.
\item New utilities that deal with patterns in the elements of BV lists.
\end{itemize}

\end{frame}

%%%%%%%%%%%%%%%%%%%%%%%%%%%%%%%%%%%%%%%%%%%%%%%%%%%%%%%%%%%%%%%%%%%%%%%%%%%%%%%%

\begin{frame}

\implibtitle

\bookpath{kestrel/crypto/ecurve}:
Elliptic curve cryptography.
\begin{itemize}
\item Extended and improved formalization of short Weierstrass curves.
\item Added formalization of twisted Edwards curves.
\item Added formalization of Montgomery curves.
\item Added formalization of birational equivalence between
      Montgomery and twisted Edwards.
\item Added formalization of Edwards BLS12 curve.
\item Added refinement of \code{pfield-squarep}.
\end{itemize}

\end{frame}

%%%%%%%%%%%%%%%%%%%%%%%%%%%%%%%%%%%%%%%%%%%%%%%%%%%%%%%%%%%%%%%%%%%%%%%%%%%%%%%%

\begin{frame}

\implibtitle

\bookpath{kestrel/ethereum}:
Ethereum library.
\begin{itemize}
\item A new sub-library for the Semaphore gadget.  Includes various
specifications and proofs of Semaphore-related R1CSes, including a
mixing function from BLAKE2s and 3 variants of the MiMC hash function.
\item Semaphore-specialized Axe tools to lift R1CSes into logic and
verify them.
\end{itemize}

\end{frame}

%%%%%%%%%%%%%%%%%%%%%%%%%%%%%%%%%%%%%%%%%%%%%%%%%%%%%%%%%%%%%%%%%%%%%%%%%%%%%%%%

\begin{frame}

\implibtitle

\bookpath{kestrel/event-macros}:
Tools for event macros.
\begin{itemize}
\item Added utilities to create events from structured information.
\item Added utility to set up a more controlled proof environment
      for generating proofs designed to never fail.
\item Other improvements.
\end{itemize}

\end{frame}

%%%%%%%%%%%%%%%%%%%%%%%%%%%%%%%%%%%%%%%%%%%%%%%%%%%%%%%%%%%%%%%%%%%%%%%%%%%%%%%%

\begin{frame}

\implibtitle

\bookpath{kestrel/file-io-light}:
Lightweight file I/O library.
\begin{itemize}
\item Various new lightweight utilities to read and write files
      (of bytes, characters, and objects).
\item New utilities to read bytes and characters from files into \code{stobj}
      arrays.
\item Reasoning support for various built-in I/O functions.
\item Utilities to check whether a file exists or is newer than a given date.
\item Proofs that bad channel names sometimes cannot occur.
\end{itemize}

\end{frame}

%%%%%%%%%%%%%%%%%%%%%%%%%%%%%%%%%%%%%%%%%%%%%%%%%%%%%%%%%%%%%%%%%%%%%%%%%%%%%%%%

\begin{frame}

\implibtitle

\bookpath{kestrel/fty}:
Fixtype library extensions in the Kestrel books.
\begin{itemize}
\item Mutual recursion (i.e.\ \code{deftypes})
      now supports \code{defset} and \code{defomap}.
\item Added a macro \code{defsubtype} for subtypes of existing fixtypes.
\item Added a macro \code{defresult} for result types,
      i.e.\ unions of good results and error results
      (similar to Rust's \code{Result} type).
\item Added several common fixtypes.
\end{itemize}

\end{frame}

%%%%%%%%%%%%%%%%%%%%%%%%%%%%%%%%%%%%%%%%%%%%%%%%%%%%%%%%%%%%%%%%%%%%%%%%%%%%%%%%

\begin{frame}

\implibtitle

\bookpath{kestrel/helpers}:
Proof helpers
\begin{itemize}
\item Draft tool for auto-generating return type theorems.
\item Rudimentary tools to discover simple proofs.
\item Rudimentary tools to improve existing books.
\item Tools for processing book dependency info.
\end{itemize}

\end{frame}

%%%%%%%%%%%%%%%%%%%%%%%%%%%%%%%%%%%%%%%%%%%%%%%%%%%%%%%%%%%%%%%%%%%%%%%%%%%%%%%%

\begin{frame}

\implibtitle

\bookpath{kestrel/java}:
Models, proofs, and tools for Java.
\begin{itemize}
\item AIJ has been optimized,
      and extended with more Java implementations of ACL2 built-in functions.
\item ATJ has been extended and improved
      (see the paper on ATJ at this Workshop).
\item Extended the formalization of (some aspects of) the Java language.
\end{itemize}

\end{frame}

%%%%%%%%%%%%%%%%%%%%%%%%%%%%%%%%%%%%%%%%%%%%%%%%%%%%%%%%%%%%%%%%%%%%%%%%%%%%%%%%

\begin{frame}

\implibtitle

\bookpath{kestrel/library-wrappers}:
Library Wrappers.
\begin{itemize}
\item A new, robust variant of \code{make-flag} (sped up some proofs by 100x
      using a custom clause processor).
\end{itemize}

\end{frame}

%%%%%%%%%%%%%%%%%%%%%%%%%%%%%%%%%%%%%%%%%%%%%%%%%%%%%%%%%%%%%%%%%%%%%%%%%%%%%%%%

\begin{frame}

\implibtitle

\bookpath{kestrel/lists-light}:
Lightweight lists library.
\begin{itemize}
\item Many new rules and congruences.
\item New books about the functions \code{subsequencep},
      \code{subsequencep-equal}, \code{last-elem}, \code{subrange},
      \code{update-subrange}, \code{finalcdr}, \code{all-equal\$},
      \code{all-eql\$}, \code{all-same}, \code{all-same-eql},
      \code{add-to-end}, \code{first-non-member}, \code{group}
      and \code{ungroup} (for splitting and flattening),
      \code{count-occs}, \code{prefixp}, \code{len-at-least},
      \code{remove-equal}, \code{remove-duplicates-equal}, \code{find-index},
      \code{remove-nth}, \code{make-list-ac}, \code{resize-list}, and
      \code{replace-item}.
\item New book about functions that treat lists like sets.
\end{itemize}

\end{frame}

%%%%%%%%%%%%%%%%%%%%%%%%%%%%%%%%%%%%%%%%%%%%%%%%%%%%%%%%%%%%%%%%%%%%%%%%%%%%%%%%

\begin{frame}

\implibtitle

\bookpath{kestrel/prime-fields}:
Prime fields library.
\begin{itemize}
\item Many new/improved rules.
\item Rules for recogizing R1CS gadgets.
\item \code{bind-free} rules for canceling addends and moving negations.
\end{itemize}

\end{frame}

%%%%%%%%%%%%%%%%%%%%%%%%%%%%%%%%%%%%%%%%%%%%%%%%%%%%%%%%%%%%%%%%%%%%%%%%%%%%%%%%

\begin{frame}

\implibtitle

\bookpath{kestrel/soft}:
Second-Order Functions and Theorems (SOFT).
\begin{itemize}
\item Added macro \code{defsoft} to record already introduced functions
      into the SOFT table for possible later instantiation.
\item Added macros \code{define2}, \code{defund-sk2}, \code{define-sk2}
      as second-order versions of the existing macros.
\item Added macro \code{defequal} to introduce second-order equalities.
\end{itemize}

\end{frame}

%%%%%%%%%%%%%%%%%%%%%%%%%%%%%%%%%%%%%%%%%%%%%%%%%%%%%%%%%%%%%%%%%%%%%%%%%%%%%%%%

\begin{frame}

\implibtitle

\bookpath{kestrel/std}:
Standard library extensions in the Kestrel books.
\begin{itemize}
\item Added several system utilities.
\item Added macro \code{defund-sk} that disables function and theorem.
\item Added macros \code{defmapping}, \code{definj}, \code{defsurj}
      to introduce and verify mappings between predicates.
\item Added macro \code{tuple} to mimic \code{mv} return specifiers
      inside components of \code{mv} return specifiers
      (particularly, the value component of error triples).
\item Added macro \code{defmin-int} to declaratively define
      the minimum of a (possibly infinite) set of integers.
\end{itemize}

\end{frame}

%%%%%%%%%%%%%%%%%%%%%%%%%%%%%%%%%%%%%%%%%%%%%%%%%%%%%%%%%%%%%%%%%%%%%%%%%%%%%%%%

\begin{frame}

\implibtitle

\bookpath{kestrel/utilities}:
Utilities in the Kestrel books (1).
\begin{itemize}
\item Added macro \code{checkpoint-list},
  which provides a programmatic, flexible interface
  to the key checkpoint information.
\item A new ACL2 Lint tool can detect common ACL2 errors and suggest
  improvements to functions and theorems.  Led to quite
  a few fixes in the Community books.
\item Reasoning about I/O channels has been improved.
\item New utilities support computing a constant using make-event,
  reading a value from a file into a \code{defconst}, and printing constants
  nicely.
\item A new tool, \code{bind-from-rules} can bind free variables in rules by
  searching existing rules.
\item Various improvements have been made to \code{defopeners} (and it now
  subsumes \code{defopeners-mut-rec}).
\end{itemize}

\end{frame}

%%%%%%%%%%%%%%%%%%%%%%%%%%%%%%%%%%%%%%%%%%%%%%%%%%%%%%%%%%%%%%%%%%%%%%%%%%%%%%%%

\begin{frame}

\implibtitle

\bookpath{kestrel/utilities}:
Utilities in the Kestrel books (2).
\begin{itemize}
\item A new data structure, string trees, can efficiently represent a
  sequence of strings (e.g., for writing to a file).
\item A new tool supports polarity-based rewriting, whereby a term can
  be either strengthened or weakened depending on whether it is an
  assumption or a conclusion.
\item New sorting utilities, including \code{split-list-fast},
  \code{merge-sort-generic}, and \code{defmergesort}.
\item New XDOC constructors, e.g., one that creates XDOC paragraphs
  from blocks of text separated by blank lines.
\item A new tool, \code{gen-xdoc-for-file}, to generate XDOC topics
  for every event in a file by extracting the relevant lines of code
  (the definition and any immediately preceding or following comment
  lines).
\item Helpful wrappers for XDOC archiving utilities.
\end{itemize}

\end{frame}

%%%%%%%%%%%%%%%%%%%%%%%%%%%%%%%%%%%%%%%%%%%%%%%%%%%%%%%%%%%%%%%%%%%%%%%%%%%%%%%%

\begin{frame}

\implibtitle

\bookpath{kestrel/utilities}:
Utilities in the Kestrel books (3).
\begin{itemize}
\item New utilities about event forms, making fresh names,
  manipulating hints, building simple list structures, dealing with
  quoted entities, building strings, checking whether a symbol has
  properties, dealing with runes, parsing options, processing keyword
  args, and recognizing legal variable names.
\item New utilities about redundant events, guard-holders, ruler-extenders,
  \code{let}s/\code{lambda}s, worlds, clause identifiers,
  \code{progn}, unification, dependencies, ensuring rules are known,
  quieting \code{make-event}, processing \code{defun} and \code{defthm} forms,
  processing \code{declare}s, the ACL2 \code{state}, \code{system-books-dir},
  fixing functions, \code{acl2-count}, \code{make-ord}, \code{coerce},
  \code{map-symbol-name}, tuples, \code{myif},\code{mv-nth},
  \code{explode-nonnegative-integer}, \code{explode-atom},
  \code{intern-in-package-of-symbol}, supporting functions,
  constant names, \code{nat-to-string}, and \code{binary-pack}.
\end{itemize}

\end{frame}

%%%%%%%%%%%%%%%%%%%%%%%%%%%%%%%%%%%%%%%%%%%%%%%%%%%%%%%%%%%%%%%%%%%%%%%%%%%%%%%%

\begin{frame}

\implibtitle

\bookpath{kestrel/utilities}:
Utilities in the Kestrel books (4).
\begin{itemize}
\item New tool \code{defstobj+}: Drop-in replacement for
  \code{defstobj} that disables functions and proves many rules
  (scalar and array fields only).
\item New utility \code{with-local-stobjs} (extends \code{with-local-stobj}
  to support multiple stobjs)
\item Draft of a tool to specialize theorems.
\item New \code{defcalculation} tool for proofs that chain equalities.
\item New books about \code{assoc-keyword}, theory-invariants,
  \code{chk-length-and-keys}, \code{member-symbol-name}, arities,
  negation, \code{logic-termp}, messages, reconstructing macro calls,
  \code{defun} and \code{mutual-recursion} forms, macro args,
  \code{digit-to-char}, finding where a name was introduced, making
  lists of symbols, and manipulating conjuncts/disjuncts.
\end{itemize}

\end{frame}

%%%%%%%%%%%%%%%%%%%%%%%%%%%%%%%%%%%%%%%%%%%%%%%%%%%%%%%%%%%%%%%%%%%%%%%%%%%%%%%%


\begin{frame}

\implibtitle

\bookpath{kestrel/utilities}:
Utilities in the Kestrel books (5).
\begin{itemize}
\item New utilities about imported symbols, format strings,
  printing, translation, manipulating terms, invariant risk,
  submitting events to ACL2, creating temp dirs, process IDs,
  usernames, calling scripts, macroexpansion and translation, and asserts.
\item New or improved utilities about \code{:program} mode,
  \code{prove\$}, \code{directed-untranslate}, ignores, translation
  (tolerating ignored vars), tables, symbol creation, \code{disjoin},
  adding documentation to macros (\code{defmacrodoc}), verifying
  guards, and non-normalized names.
\end{itemize}

\end{frame}

%%%%%%%%%%%%%%%%%%%%%%%%%%%%%%%%%%%%%%%%%%%%%%%%%%%%%%%%%%%%%%%%%%%%%%%%%%%%%%%%




\begin{frame}

\implibtitle

\bookpath{nonstd}:
Non-standard analysis.
\begin{itemize}
\item Formalization of Banach-Tarski paradox in ACL2(r),
      at \code{nonstd/nsa/Banach-Tarski/}.
\item Properties of 3D rotations
      using the \code{array2p} data structure in ACL2(r),
      at \code{nonstd/nsa/Banach-Tarski/rotations.lisp}.
\item Integration by substitution in ACL2(r),
      and proof of the area of a circle in ACL2(r),
      at \code{nonstd/integrals/u-substitution.lisp}.
\end{itemize}

\end{frame}

%%%%%%%%%%%%%%%%%%%%%%%%%%%%%%%%%%%%%%%%%%%%%%%%%%%%%%%%%%%%%%%%%%%%%%%%%%%%%%%%

\begin{frame}

\implibtitle

\bookpath{projects/apply}:
\code{apply\$} and \code{loop\$} tools.
\begin{itemize}
\item Replaced \code{top.lisp} with:
      \begin{itemize}
      \item \code{apply.lisp}, to reason about \code{apply\$}.
      \item \code{loop.lisp}, to reason about \code{loop\$}
            (this also includes \code{apply.lisp}).
      \item \code{top.lisp},
            which includes \code{apply.lisp} and \code{loop.lisp},
            and is thus the same as \code{loop.lisp}.
      \end{itemize}
\item Made the inclusion of certain supporting books local.
\end{itemize}

\end{frame}

%%%%%%%%%%%%%%%%%%%%%%%%%%%%%%%%%%%%%%%%%%%%%%%%%%%%%%%%%%%%%%%%%%%%%%%%%%%%%%%%

\begin{frame}

\implibtitle

\bookpath{projects/rac}:
Restricted Algorithmic C (RAC).
\begin{itemize}
\item The \code{tuple} template can have up to {\em eight} arguments.
\item Support the \code{struct} data type.
\item Report more detailed error messages.
\end{itemize}

\end{frame}

%%%%%%%%%%%%%%%%%%%%%%%%%%%%%%%%%%%%%%%%%%%%%%%%%%%%%%%%%%%%%%%%%%%%%%%%%%%%%%%%

\begin{frame}

\implibtitle

\bookpath{projects/x86isa}:
X86ISA, the formal model of the x86 Instruction Set Architecture.
\begin{itemize}
\item Simplified the treatment of CPUID features.
\item Added the \code{MOVD} and \code{MOVQ} instruction variants
      that move data from/to the XMM registers.
\item Simplified state definition by using
      \code{centaur/bigmem} and \code{centaur/defrstobj2}. See
      \code{:doc x86isa-state-history} for details.
\end{itemize}

\end{frame}

%%%%%%%%%%%%%%%%%%%%%%%%%%%%%%%%%%%%%%%%%%%%%%%%%%%%%%%%%%%%%%%%%%%%%%%%%%%%%%%%

\begin{frame}

\implibtitle

\bookpath{rtl}:
Register-transfer logic library.
\begin{itemize}
\item Added a signed version of the radix-4 Booth encoding algorithm.
\item A new section of \code{books/rtl/rel11/lib/round.lisp} on
      underflow detection, corresponding to Section 6.7 of ``Formal
      Verification of Floating-Point Hardware Design'', 2nd edition.
\end{itemize}

\end{frame}

%%%%%%%%%%%%%%%%%%%%%%%%%%%%%%%%%%%%%%%%%%%%%%%%%%%%%%%%%%%%%%%%%%%%%%%%%%%%%%%%

\begin{frame}

\implibtitle

\bookpath{std}:
Standard library.
\begin{itemize}
\item Added macros \code{add-io-pairs} and \code{merge-io-pairs}
      to speed up execution using verified input/output pairs.
      The idea is if you have a simple function that is infeasible to
      execute but feasible to verify on a concrete I/O pair,
      then that proof can be used to memoize the function to make execution
      feasible on that input, without changing its definition or any callers.
\item Moved macro \code{define-sk} from the Kestrel books.
\item Added several typed alists.
\end{itemize}

\end{frame}

%%%%%%%%%%%%%%%%%%%%%%%%%%%%%%%%%%%%%%%%%%%%%%%%%%%%%%%%%%%%%%%%%%%%%%%%%%%%%%%%

\begin{frame}

\implibtitle

\bookpath{tools}:
Miscellaneous tools.
\begin{itemize}
\item Added macro \code{rewrite\$},
      which provides a programmatic, flexible interface
      to the ACL2 rewriter.
\item Improved macro \code{prove\$},
      which provides a programmatic, flexible interface
      to the ACL2 prover.
\item Improved \code{make-flag} (support computed hints better,
      better template theorems).
\end{itemize}
\end{frame}

%%%%%%%%%%%%%%%%%%%%%%%%%%%%%%%%%%%%%%%%%%%%%%%%%%%%%%%%%%%%%%%%%%%%%%%%%%%%%%%%

\end{document}
