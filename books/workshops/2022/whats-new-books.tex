% GENERAL FORMAT OF EACH SLIDE
% ----------------------------

%\begin{frame}

%\newlibtitle or \implibtitle % for new vs. improved libraries

%\bookpath{path/to/the/library}: % directory or file, use macro for font
% Short description of what this library is, 1-2 lines.
%\begin{itemize}
%\item
% A highlight about this new or improved library.
%\item
% Another highlight.
%\item
% Say if it's described in a paper at this workshop,
% in which case, there may be little to say here.
%\end{itemize}

%\separation % to separate the next entry, if any

%\bookpath{another/path}: % as above

%\end{frame}

% ORDER OF THE SLIDES AND ENTRIES
% -------------------------------

% First all the new libraries, in alphabetical order.

% Then all the improved libraries, in alphabetical order.

% Then one or more slides with additional contributions.

% Don't worry about adding your contributions in order:
% Alessandro will quickly put them in order if needed,
% and also fix any formatting issues.

% TITLE SLIDE
% -----------

% Add your name as co-author, in alphabetical order of last name.

% Add your organization, in alphabetical order.
% Renumber the \inst{n} instances as needed
% (both inside \author{...} and inside \institute{...}).

% Or ask Alessandro to do this.

% OTHER REMARKS
% -------------

% Use the \code{...} macro for code (i.e.\ fixed-width) font within the text.

% If you want to show code block snippets,
% use the verbatim environment,
% and add [fragile] just after \begin{frame}.

% For guidelines on content and level of detail,
% please see existing entries.
% The goal is to provide enough information to get possible users interested.

%%%%%%%%%%%%%%%%%%%%%%%%%%%%%%%%%%%%%%%%%%%%%%%%%%%%%%%%%%%%%%%%%%%%%%%%%%%%%%%%

\documentclass{beamer}
\usetheme{Singapore}
\usecolortheme{orchid}
\usepackage{inconsolata}

\beamertemplatenavigationsymbolsempty

\setbeamertemplate{footline}[frame number]

%%%%%%%%%%%%%%%%%%%%%%%%%%%%%%%%%%%%%%%%%%%%%%%%%%%%%%%%%%%%%%%%%%%%%%%%%%%%%%%%

\newcommand{\code}[1]{\texttt{#1}}
\newenvironment{codeblock}{\begin{alltt}}{\end{alltt}}

\newcommand{\bookpath}[1]{\textbf{\code{#1}}}

\newcommand{\newlibtitle}{\frametitle{New Libraries}}
\newcommand{\implibtitle}{\frametitle{Improved Libraries}}

\newcommand{\separation}{\vspace*{1.5ex}}

%%%%%%%%%%%%%%%%%%%%%%%%%%%%%%%%%%%%%%%%%%%%%%%%%%%%%%%%%%%%%%%%%%%%%%%%%%%%%%%%

\title{What's New in the Community Books}

\subtitle{Since the ACL2-2020 Workshop}

\author{Jagadish Bapanapally\inst{8}\inst{4},
        Cuong Chau\inst{2},\\
        Alessandro Coglio\inst{1}\inst{4}\inst{5} (presenter),
        Shilpi Goel\inst{3},\\
        Matt Kaufmann\inst{7},
        Panagiotis Manolios\inst{6},\\
        Eric McCarthy\inst{4},
        J Strother Moore\inst{7},
        David Russinoff\inst{2}}

\institute{\inst{1}Aleo,
           \inst{2}ARM,
           \inst{3}Centaur,
           \inst{4}Kestrel Institute,
           \inst{5}Kestrel Technology,\\
           \inst{6}Northeatern University,
           \inst{7}University of Texas at Austin (retired),
           \inst{8}University of Wyoming}

\date{ACL2-2022 Workshop}

%%%%%%%%%%%%%%%%%%%%%%%%%%%%%%%%%%%%%%%%%%%%%%%%%%%%%%%%%%%%%%%%%%%%%%%%%%%%%%%%

\begin{document}

%%%%%%%%%%%%%%%%%%%%%%%%%%%%%%%%%%%%%%%%%%%%%%%%%%%%%%%%%%%%%%%%%%%%%%%%%%%%%%%%

\frame[noframenumbering,plain]{\titlepage}

%%%%%%%%%%%%%%%%%%%%%%%%%%%%%%%%%%%%%%%%%%%%%%%%%%%%%%%%%%%%%%%%%%%%%%%%%%%%%%%%

\begin{frame}[fragile]

\newlibtitle

\bookpath{centaur/bigmem}:
A library that defines a big array (up to 2$^{64}$ bytes).
\begin{itemize}
\item Offers the reasoning efficiency of records.
\item Offers efficient execution via nested stobjs containing resizable arrays.
\end{itemize}

\end{frame}

%%%%%%%%%%%%%%%%%%%%%%%%%%%%%%%%%%%%%%%%%%%%%%%%%%%%%%%%%%%%%%%%%%%%%%%%%%%%%%%%

\begin{frame}[fragile]

\newlibtitle

\bookpath{kestrel/acl2pl}:
Preliminary model of the ACL2 programming (not logical) language.
\begin{itemize}
\item Abstract syntax consisting of translated terms, functions, and packages.
\item Small-step evaluation semantics.
\item Program-mode interpreter.
\item Lifter from packages and functions in the world to the formal model.
\end{itemize}

\end{frame}

%%%%%%%%%%%%%%%%%%%%%%%%%%%%%%%%%%%%%%%%%%%%%%%%%%%%%%%%%%%%%%%%%%%%%%%%%%%%%%%%

\begin{frame}

\newlibtitle

\bookpath{kestrel/c}:
Models, proofs, and tools for C (described in a paper at this workshop).
\begin{itemize}
\item ATC, the C code generator for ACL2.
\item Deep embedding of C in ACL2.
\item Shallow embedding of C in ACL2.
\end{itemize}

\end{frame}

%%%%%%%%%%%%%%%%%%%%%%%%%%%%%%%%%%%%%%%%%%%%%%%%%%%%%%%%%%%%%%%%%%%%%%%%%%%%%%%%

\begin{frame}

\newlibtitle

\bookpath{kestrel/crypto/pfcs}:
Prime Field Constraint Systems (PFCS).
\begin{itemize}
\item Generalization of Rank-1 Constraint Systems (R1CS).
\item Formal syntax and semantics.
\item Preliminary reasoning support.
\item Useful in zero-knowledge cryptographic proofs.
\end{itemize}

\end{frame}

%%%%%%%%%%%%%%%%%%%%%%%%%%%%%%%%%%%%%%%%%%%%%%%%%%%%%%%%%%%%%%%%%%%%%%%%%%%%%%%%

\begin{frame}

\newlibtitle

\bookpath{kestrel/isar}:
Tools for Isar-style proofs in ACL2.
\begin{itemize}
\item Isar = Intelligible semi-automated reasoning
      (proof language of the Isabelle theorem prover).
\item Human-oriented (vs.\ machine-oriented) readable proofs.
\item \code{:assume} ... \code{:let} ... \code{:derive} ... \code{:qed}.
\item Useful, for instance, in proofs involving algebraic manipulations
      that do not follow simple rewriting directions.
\item This library is just a small start.
\end{itemize}

\end{frame}

%%%%%%%%%%%%%%%%%%%%%%%%%%%%%%%%%%%%%%%%%%%%%%%%%%%%%%%%%%%%%%%%%%%%%%%%%%%%%%%%

\begin{frame}

\newlibtitle

\bookpath{kestrel/json}:
Models and tools for JSON.
\begin{itemize}
\item Abstract syntax of JSON.
\item Converter from parser's abstract syntax to this abstract syntax.
\item Can be extended with semantics and tools.
\end{itemize}

\end{frame}

%%%%%%%%%%%%%%%%%%%%%%%%%%%%%%%%%%%%%%%%%%%%%%%%%%%%%%%%%%%%%%%%%%%%%%%%%%%%%%%%

\begin{frame}

\newlibtitle

\bookpath{kestrel/simpl-imp}:
A simple didactic programming language, Imp.
\begin{itemize}
\item Formal syntax and semantics.
\item Useful for didactic purposes.
\item Used in the literature.
\end{itemize}

\end{frame}

%%%%%%%%%%%%%%%%%%%%%%%%%%%%%%%%%%%%%%%%%%%%%%%%%%%%%%%%%%%%%%%%%%%%%%%%%%%%%%%%

\begin{frame}

\newlibtitle

\bookpath{kestrel/solidity}:
Small start towards a model of Solidity,
the main smart contract language for Ethereum.
\begin{itemize}
\item A model of (some) Solidity values.
\end{itemize}

\end{frame}

%%%%%%%%%%%%%%%%%%%%%%%%%%%%%%%%%%%%%%%%%%%%%%%%%%%%%%%%%%%%%%%%%%%%%%%%%%%%%%%%

\begin{frame}

\newlibtitle

\bookpath{kestrel/syntheto}:
Models, proofs, and tools for Syntheto (described in a paper at this workshop).
\begin{itemize}
\item Surface language for APT and ACL2.
\item Formalization of abstract syntax.
\item Formalization of static semantics.
\item Translation to ACL2 (Syntheto back end).
\end{itemize}

\end{frame}

%%%%%%%%%%%%%%%%%%%%%%%%%%%%%%%%%%%%%%%%%%%%%%%%%%%%%%%%%%%%%%%%%%%%%%%%%%%%%%%%

\begin{frame}

\newlibtitle

\bookpath{kestrel/yul}:
Models, proofs, and tools for Yul,
an intermediate language used in the Solidity compiler.
\begin{itemize}
\item Concrete syntax and parser.
\item Abstract syntax.
\item Static semantics.
\item Dynamic semantics.
\item Static soundness proof.
\item Covers all of `generic Yul'.
\item Some verified Yul-to-Yul transformations (used in the Solidity compiler).
\end{itemize}

\end{frame}

%%%%%%%%%%%%%%%%%%%%%%%%%%%%%%%%%%%%%%%%%%%%%%%%%%%%%%%%%%%%%%%%%%%%%%%%%%%%%%%%

\begin{frame}

\newlibtitle

\bookpath{kestrel/zcash}:
Models and proofs for the Zcash blockchain.
\begin{itemize}
\item Formalization of some zero-knowledge-related operations.
\item Formalization and verification of some R1CS gadgets.
\end{itemize}
\end{frame}

%%%%%%%%%%%%%%%%%%%%%%%%%%%%%%%%%%%%%%%%%%%%%%%%%%%%%%%%%%%%%%%%%%%%%%%%%%%%%%%%

\begin{frame}

\newlibtitle

\bookpath{projects/execloader}:
Binary loaders.
\begin{itemize}
\item Read in sections of ELF/Mach-O files into ACL2. An older version
      of these books used to live in the \code{x86isa} library.
\item Simplified \code{elf-reader}; ELF binary header, all section
      headers, and all section contents are now stored in the \code{elf}
      stobj. Previously, only a handful of commonly-used sections (e.g.,
      \code{.text}, \code{.data}, \code{.rodata}, etc.) were parsed.
\item Added support for getting information from ELF symbol table
      using functions \code{get-symtab-entries} and
      \code{get-label-address}.
\end{itemize}
\end{frame}

%%%%%%%%%%%%%%%%%%%%%%%%%%%%%%%%%%%%%%%%%%%%%%%%%%%%%%%%%%%%%%%%%%%%%%%%%%%%%%%%

\begin{frame}

\newlibtitle

\bookpath{std/obags}:
Ordered bags (obags).
\begin{itemize}
\item Similar to osets and omaps, for bags (i.e.\ multisets).
\item Modeled as totally (non-strictly) ordered lists.
\item Include operations and theorems.
\end{itemize}

\end{frame}

%%%%%%%%%%%%%%%%%%%%%%%%%%%%%%%%%%%%%%%%%%%%%%%%%%%%%%%%%%%%%%%%%%%%%%%%%%%%%%%%

\begin{frame}

\implibtitle

\bookpath{centaur/defrstobj2}:
Record-like stobjs.
\begin{itemize}
\item \code{defrstobj2} can now be used to define stobjs with child
      stobj fields, i.e., fields based on another stobj.
\end{itemize}

\end{frame}

%%%%%%%%%%%%%%%%%%%%%%%%%%%%%%%%%%%%%%%%%%%%%%%%%%%%%%%%%%%%%%%%%%%%%%%%%%%%%%%%

\begin{frame}

\implibtitle

\bookpath{kestrel/abnf}:
Augmented Backus-Naur Form (ABNF).
\begin{itemize}
\item Refactored to move parser verification proof to separate file from parser.
\item Refactored to collect parsing primitives usable in other parsers.
\item Added preliminary parsing generation tools.
\end{itemize}

\end{frame}

%%%%%%%%%%%%%%%%%%%%%%%%%%%%%%%%%%%%%%%%%%%%%%%%%%%%%%%%%%%%%%%%%%%%%%%%%%%%%%%%

\begin{frame}

\implibtitle

\bookpath{acl2s}: ACL2s Sedan.
\begin{itemize}
\item Added \code{properties} book with \code{definec}-like syntax
  supporting property-based design, testing and verification.
\item Various improvements to \code{definec} and \code{defunc}
  regarding performance, debugging support, extensions.
\item Updated utilities, such as \code{acl2s::match} macro, for pattern matching.
\item More polymorphic support, built-in types, alias types, etc.
  in \code{defdata}.
\item Improvements for counterexample generations with \code{cgen},
  including using fixers in ACL2s.
\end{itemize}

\end{frame}

%%%%%%%%%%%%%%%%%%%%%%%%%%%%%%%%%%%%%%%%%%%%%%%%%%%%%%%%%%%%%%%%%%%%%%%%%%%%%%%%

\begin{frame}

\implibtitle

\bookpath{kestrel/apt}:
Automated Program Transformations (APT).
\begin{itemize}
\item Added new \code{schemalg} transformation to apply algorithm schemas,
      currently supporting divide-and-conquer schemas.
\item Added new \code{solve} transformation to attempt to solve a specification,
      currently supporting the ACL2 and Axe rewriters as solvers.
\item Added new \code{expdata} transformation to refine data types
      where each instance of the old data may be represented byte
      multiple instances of the new data (i.e\ not isomorphic).
\item Improved and extended existing
      \code{isodata}, \code{restrict}, \code{simplify}, and \code{tailrec}
      transformations.
\end{itemize}

\end{frame}

%%%%%%%%%%%%%%%%%%%%%%%%%%%%%%%%%%%%%%%%%%%%%%%%%%%%%%%%%%%%%%%%%%%%%%%%%%%%%%%%

\begin{frame}

\implibtitle

\bookpath{kestrel/bitcoin}:
Bitcoin.
\begin{itemize}
\item Added formalization of the Bech32 and Bech32m checksummed base32 formats
      used to encode addresses in Segwit.
\end{itemize}

\end{frame}

%%%%%%%%%%%%%%%%%%%%%%%%%%%%%%%%%%%%%%%%%%%%%%%%%%%%%%%%%%%%%%%%%%%%%%%%%%%%%%%%

\begin{frame}

\implibtitle

\bookpath{build}:
Build system.
\begin{itemize}
\item Now \code{cert.pl} makes use of useless runes just as \code{make} does.
\item Now \code{make} of the community books also writes book dependency
      information in S-expression form.
\end{itemize}

\end{frame}

%%%%%%%%%%%%%%%%%%%%%%%%%%%%%%%%%%%%%%%%%%%%%%%%%%%%%%%%%%%%%%%%%%%%%%%%%%%%%%%%

\begin{frame}

\implibtitle

\bookpath{kestrel/crypto/ecurve}:
Elliptic curve cryptography.
\begin{itemize}
\item Extended and improved formalization of short Weierstrass curves.
\item Added formalization of twisted Edwards curves.
\item Added formalization of Montgomery curves.
\item Added formalization of birational equivalence between
      Montgomery and twisted Edwards.
\item Added formalization of Edwards BLS12 curve.
\end{itemize}

\end{frame}

%%%%%%%%%%%%%%%%%%%%%%%%%%%%%%%%%%%%%%%%%%%%%%%%%%%%%%%%%%%%%%%%%%%%%%%%%%%%%%%%

\begin{frame}

\implibtitle

\bookpath{kestrel/event-macros}:
Tools for event macros.
\begin{itemize}
\item Added utilities to create events from structured information.
\item Added utility to set up a more controlled proof environment
      for generating proofs designed to never fail.
\item Other improvements.
\end{itemize}

\end{frame}

%%%%%%%%%%%%%%%%%%%%%%%%%%%%%%%%%%%%%%%%%%%%%%%%%%%%%%%%%%%%%%%%%%%%%%%%%%%%%%%%

\begin{frame}

\implibtitle

\bookpath{kestrel/fty}:
Fixtype library extensions in the Kestrel books.
\begin{itemize}
\item Mutual recursion (i.e.\ \code{deftypes})
      now supports \code{defset} and \code{defomap}.
\item Added a macro \code{defsubtype} for subtypes of existing fixtypes.
\item Added a macro \code{defresult} for result types,
      i.e.\ unions of good results and error results
      (similar to Rust's \code{Result} type).
\item Added several common fixtypes.
\end{itemize}

\end{frame}

%%%%%%%%%%%%%%%%%%%%%%%%%%%%%%%%%%%%%%%%%%%%%%%%%%%%%%%%%%%%%%%%%%%%%%%%%%%%%%%%

\begin{frame}

\implibtitle

\bookpath{kestrel/java}:
Models, proofs, and tools for Java.
\begin{itemize}
\item AIJ has been optimized,
      and extended with more Java implementations of ACL2 built-in functions.
\item ATJ has been extended and improved
      (see the paper on ATJ at this Workshop).
\item Extended the formalization of (some aspects of) the Java language.
\end{itemize}

\end{frame}

%%%%%%%%%%%%%%%%%%%%%%%%%%%%%%%%%%%%%%%%%%%%%%%%%%%%%%%%%%%%%%%%%%%%%%%%%%%%%%%%

\begin{frame}

\implibtitle

\bookpath{kestrel/number-theory}:
Number theory library.
\begin{itemize}
\item Verified Tonelli-Shanks algorithm for modular square roots.
\end{itemize}
\end{frame}

%%%%%%%%%%%%%%%%%%%%%%%%%%%%%%%%%%%%%%%%%%%%%%%%%%%%%%%%%%%%%%%%%%%%%%%%%%%%%%%%

\begin{frame}

\implibtitle

\bookpath{kestrel/soft}:
Second-Order Functions and Theorems (SOFT).
\begin{itemize}
\item Added macro \code{defsoft} to record already introduced functions
      into the SOFT table for possible later instantiation.
\item Added macros \code{define2}, \code{defund-sk2}, \code{define-sk2}
      as second-order versions of the existing macros.
\item Added macro \code{defequal} to introduce second-order equalities.
\end{itemize}

\end{frame}

%%%%%%%%%%%%%%%%%%%%%%%%%%%%%%%%%%%%%%%%%%%%%%%%%%%%%%%%%%%%%%%%%%%%%%%%%%%%%%%%

\begin{frame}

\implibtitle

\bookpath{kestrel/std}:
Standard library extensions in the Kestrel books.
\begin{itemize}
\item Added several system utilities.
\item Added macro \code{defund-sk} that disables function and theorem.
\item Added macros \code{defmapping}, \code{definj}, \code{defsurj}
      to introduce and verify mappings between predicates.
\item Added macro \code{tuple} to mimic \code{mv} return specifiers
      inside components of \code{mv} return specifiers
      (particularly, the value component of error triples).
\item Added macro \code{defmin-int} to declaratively define
      the minimum of a (possibly infinite) set of integers.
\end{itemize}

\end{frame}

%%%%%%%%%%%%%%%%%%%%%%%%%%%%%%%%%%%%%%%%%%%%%%%%%%%%%%%%%%%%%%%%%%%%%%%%%%%%%%%%

\begin{frame}

\implibtitle

\bookpath{kestrel/utilities}:
Utilities in the Kestrel books.
\begin{itemize}
\item Added macro \code{checkpoint-list},
      which provides a programmatic, flexible interface
      to the key checkpoint information.
\end{itemize}

\end{frame}

%%%%%%%%%%%%%%%%%%%%%%%%%%%%%%%%%%%%%%%%%%%%%%%%%%%%%%%%%%%%%%%%%%%%%%%%%%%%%%%%

\begin{frame}

\implibtitle

\bookpath{nonstd}:
Non-standard analysis.
\begin{itemize}
\item Formalization of Banach-Tarski paradox in ACL2(r),
      at \code{nonstd/nsa/Banach-Tarski/}.
\item Properties of 3D rotations
      using the \code{array2p} data structure in ACL2(r),
      at \code{nonstd/nsa/Banach-Tarski/rotations.lisp}.
\item Integration by substitution in ACL2(r),
      and proof of the area of a circle in ACL2(r),
      at \code{nonstd/integrals/u-substitution.lisp}.
\end{itemize}

\end{frame}

%%%%%%%%%%%%%%%%%%%%%%%%%%%%%%%%%%%%%%%%%%%%%%%%%%%%%%%%%%%%%%%%%%%%%%%%%%%%%%%%

\begin{frame}

\implibtitle

\bookpath{projects/apply}:
\code{apply\$} and \code{loop\$} tools.
\begin{itemize}
\item Replaced \code{top.lisp} with:
      \begin{itemize}
      \item \code{apply.lisp}, to reason about \code{apply\$}.
      \item \code{loop.lisp}, to reason about \code{loop\$}
            (this also includes \code{apply.lisp}).
      \item \code{top.lisp},
            which includes \code{apply.lisp} and \code{loop.lisp},
            and is thus the same as \code{loop.lisp}.
      \end{itemize}
\item Made the inclusion of certain supporting books local.
\end{itemize}

\end{frame}

%%%%%%%%%%%%%%%%%%%%%%%%%%%%%%%%%%%%%%%%%%%%%%%%%%%%%%%%%%%%%%%%%%%%%%%%%%%%%%%%

\begin{frame}

\implibtitle

\bookpath{projects/rac}:
Restricted Algorithmic C (RAC).
\begin{itemize}
\item The \code{tuple} template can have up to {\em eight} arguments.
\item Support the \code{struct} data type.
\item Report more detailed error messages.
\end{itemize}

\end{frame}

%%%%%%%%%%%%%%%%%%%%%%%%%%%%%%%%%%%%%%%%%%%%%%%%%%%%%%%%%%%%%%%%%%%%%%%%%%%%%%%%

\begin{frame}

\implibtitle

\bookpath{rtl}:
Register-transfer logic library.
\begin{itemize}
\item Added a signed version of the radix-4 Booth encoding algorithm.
\item A new section of \code{books/rtl/rel11/lib/round.lisp} on
      underflow detection, corresponding to Section 6.7 of ``Formal
      Verification of Floating-Point Hardware Design'', 2nd edition.
\end{itemize}

\end{frame}

%%%%%%%%%%%%%%%%%%%%%%%%%%%%%%%%%%%%%%%%%%%%%%%%%%%%%%%%%%%%%%%%%%%%%%%%%%%%%%%%

\begin{frame}

\implibtitle

\bookpath{projects/x86isa}:
X86ISA, the formal model of the x86 Instruction Set Architecture.
\begin{itemize}
\item Simplified the treatment of CPUID features.
\item Added the \code{MOVD} and \code{MOVQ} instruction variants
      that move data from/to the XMM registers.
\item Simplified state definition by using
      \code{centaur/bigmem} and \code{centaur/defrstobj2}. See
      \code{:doc x86isa-state-history} for details.
\end{itemize}

\end{frame}

%%%%%%%%%%%%%%%%%%%%%%%%%%%%%%%%%%%%%%%%%%%%%%%%%%%%%%%%%%%%%%%%%%%%%%%%%%%%%%%%

\begin{frame}

\implibtitle

\bookpath{std}:
Standard library.
\begin{itemize}
\item Added macro \code{add-io-pairs} to speed up execution
      using verified input/output pairs.
\item Moved macro \code{define-sk} from the Kestrel books.
\item Added several typed alists.
\end{itemize}

\end{frame}

%%%%%%%%%%%%%%%%%%%%%%%%%%%%%%%%%%%%%%%%%%%%%%%%%%%%%%%%%%%%%%%%%%%%%%%%%%%%%%%%

\begin{frame}

\implibtitle

\bookpath{tools}: Miscellaneous tools.
\begin{itemize}
\item Added macro \code{rewrite\$},
      which provides a programmatic, flexible interface
      to the ACL2 rewriter.
\item Improved macro \code{prove\$},
      which provides a programmatic, flexible interface
      to the ACL2 prover.
\end{itemize}
\end{frame}

%%%%%%%%%%%%%%%%%%%%%%%%%%%%%%%%%%%%%%%%%%%%%%%%%%%%%%%%%%%%%%%%%%%%%%%%%%%%%%%%

\begin{frame}

\implibtitle

\bookpath{centaur/fgl}: FGL symbolic execution engine.
\begin{itemize}
\item Added incremental-minimize/maximize and minimize/maximize-ratio tools
\end{itemize}
\end{frame}

%%%%%%%%%%%%%%%%%%%%%%%%%%%%%%%%%%%%%%%%%%%%%%%%%%%%%%%%%%%%%%%%%%%%%%%%%%%%%%%%

\begin{frame}

\implibtitle

\bookpath{centaur/sv}: Hardware verification backend.
\begin{itemize}
\item New flow for producing a symbolic unrolling (SVTV)
\item More complete logical story for process from hierarchical design to unrolling
\item New utilities for proof (de)composition (def-svtv-override-fact)
\item Improvements to SVTV-CHASE utility
\end{itemize}
\end{frame}

%%%%%%%%%%%%%%%%%%%%%%%%%%%%%%%%%%%%%%%%%%%%%%%%%%%%%%%%%%%%%%%%%%%%%%%%%%%%%%%%

\begin{frame}

\implibtitle

\bookpath{centaur/vl}: SystemVerilog frontend.
\begin{itemize}
\item New support for SystemVerilog calls of static methods of parametrized classes
\item Improved preprocessor performance when there are lots of defines
\end{itemize}
\end{frame}

%%%%%%%%%%%%%%%%%%%%%%%%%%%%%%%%%%%%%%%%%%%%%%%%%%%%%%%%%%%%%%%%%%%%%%%%%%%%%%%%

\end{document}
